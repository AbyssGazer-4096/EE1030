\let\negmedspace\undefined
\let\negthickspace\undefined
\documentclass[journal]{IEEEtran}
\usepackage[a5paper, margin=10mm, onecolumn]{geometry}
%\usepackage{lmodern} % Ensure lmodern is loaded for pdflatex
\usepackage{tfrupee} % Include tfrupee package

\setlength{\headheight}{1cm} % Set the height of the header box
\setlength{\headsep}{0mm}  % Set the distance between the header box and the top of the text

\usepackage{gvv-book}
\usepackage{gvv}
\usepackage{cite}
\usepackage{amsmath,amssymb,amsfonts,amsthm}
\usepackage{algorithmic}
\usepackage{graphicx}
\usepackage{textcomp}
\usepackage{xcolor}
\usepackage{txfonts}
\usepackage{listings}
\usepackage{enumitem}
\usepackage{mathtools}
\usepackage{gensymb}
\usepackage{comment}
\usepackage[breaklinks=true]{hyperref}
\usepackage{tkz-euclide} 
\usepackage{listings}
% \usepackage{gvv}                                        
\def\inputGnumericTable{}                                 
\usepackage[latin1]{inputenc}                                
\usepackage{color}
\usepackage{array}                                            
\usepackage{longtable}                                       
\usepackage{calc}                                             
\usepackage{multirow}                                         
\usepackage{hhline}                                           
\usepackage{ifthen}                                           
\usepackage{lscape}
\usepackage{tikz}
\usetikzlibrary{patterns}
\begin{document}

\bibliographystyle{IEEEtran}
\vspace{3cm}

\title{\textbf{ASSIGNMENT-5\\GATE ME-2019}}
\author{EE24BTECH11019 - DWARAK A}
\maketitle

\bigskip

\renewcommand{\thefigure}{\theenumi}
\renewcommand{\thetable}{\theenumi}

\begin{enumerate}

\subsection*{Q.1 to Q.25 carry one mark each.}

    %17
    \item The table presents the demand of a product. By simple three-months moving average method, the demand-forecast of the product for the month of September is
        \begin{table}[!ht]
            \centering
            \begin{tabular}{|l|l|l|}
    \hline
    Job & Processing time (days) & Due date (days) \\
    \hline
    A   & 3                      & 8               \\
    \hline
    B   & 7                      & 16              \\
    \hline
    C   & 4                      & 4               \\
    \hline
    D   & 9                      & 18              \\
    \hline
    E   & 5                      & 17              \\
    \hline
    F   & 13                     & 19              \\
    \hline
\end{tabular}

        \end{table}
        \begin{enumerate}
            \item $490$
            \item $510$
            \item $530$
            \item $536.67$
        \end{enumerate}

    %18
    \item Evaluation of $\int\limits_2^4x^3dx$ using a 2-equal-segment trapezoidal rule gives a value of \underline{\hspace{1cm}}
    
    %19
    \item A block of mass $10 kg$ rests on a horizontal floor. The acceleration due to gravity is $9.81 m/s^2$. The coefficient of static friction between the floor and the block is $0.2$. A horizontal force of $10 N$ is applied on the block as shown in the figure. The magnitude of force of friction (in $N$) on the block is \underline{\hspace{1cm}}
        \begin{figure}[!ht]
            \centering
            \resizebox{0.7\textwidth}{!}{
\begin{circuitikz}
\tikzstyle{every node}=[font=\LARGE]
\draw [ line width=1pt ] (5,13.75) rectangle (13.75,12.5);
\draw [line width=0.8pt, short] (3.75,12.5) -- (15,12.5);
\draw [line width=0.8pt, short] (3.75,12.5) -- (3.25,12);
\draw [line width=0.8pt, short] (5,12.5) -- (4.5,12);
\draw [line width=0.8pt, short] (6.25,12.5) -- (5.75,12);
\draw [line width=0.8pt, short] (7.5,12.5) -- (7,12);
\draw [line width=0.8pt, short] (8.75,12.5) -- (8.25,12);
\draw [line width=0.8pt, short] (10,12.5) -- (9.5,12);
\draw [line width=0.8pt, short] (11.25,12.5) -- (10.75,12);
\draw [line width=0.8pt, short] (12.5,12.5) -- (12,12);
\draw [line width=0.8pt, short] (13.75,12.5) -- (13.25,12);
\draw [line width=0.8pt, short] (15,12.5) -- (14.5,12);
\node [font=\LARGE] at (9.25,13.25) {$10 kg$};
\draw [line width=0.8pt, ->, >=Stealth] (13.75,13.25) -- (18.75,13.25)node[pos=0.5,above, fill=white]{10 N};
\end{circuitikz}
}

        \end{figure}

    %20
    \item  A cylindrical rod of diameter $10 mm$ and length $1.0 m$ is fixed at one end. The other end is twisted by an angle of $10^\degree$ by applying a torque. If the maximum shear strain in the rod is $p\times10^{-3}$, then $p$ is equal to \underline{\hspace{1cm}} (round off to two decimal places).

    %21
    \item A solid cube of side $1 m$ is kept at a room temperature of $32 ^\degree C$. The coefficient of linear thermal expansion of the cube material is $1\times 10^{-5}/{}^{\degree} C$ and the bulk modulus is $200 GPa$. If the cube is constrained all around and heated uniformly to $42 ^\degree C$, then the magnitude of volumetric (mean) stress (in $MPa$) induced due to heating is \underline{\hspace{1cm}}

    %22
    \item During a high cycle fatigue test, a metallic specimen is subjected to cyclic loading with a mean stress of $+140 MPa$, and a minimum stress of $-70 MPa$. The $R$-ratio (minimum stress to maximum stress) for this cyclic loading is \underline{\hspace{1cm}} (round off to one decimal place).

    %23
    \item Water flows through a pipe with a velocity given by $\vec{V} = \brak{\frac{4}{t}+x+y}\hat{j}m/s$, where $\hat{j}$ is the unit vector in the $y$ direction, $t (> 0)$ is in seconds, and $x$ and $y$ are in meters. The magnitude of total acceleration at the point $(x, y) = (1, 1)$ a $t = 2 s$ is \underline{\hspace{1cm}} $m/s^2$

    %24
    \item Air of mass $1 kg$, initially at $300 K$ and $10\,bar$, is allowed to expand isothermally till it reaches a pressure of $1\,bar$. Assuming air as an ideal gas with gas constant of $0.287 kJ/kg K$, the change in entropy of air (in $kJ/kg.K$, round off to two decimal places) is \underline{\hspace{1cm}}

    %25
    \item Consider the stress-strain curve for an ideal elastic-plastic strain hardening metal as shown in the figure. The metal was loaded in uniaxial tension starting from $\vec{O}$. Upon loading, the stress-strain curve passes through initial yield point at $\vec{P}$, and then strain hardens to point $\vec{Q}$, where the loading was stopped. From point $\vec{Q}$, the specimen was unloaded to point $\vec{R}$, where the stress is zero. If the same specimen is reloaded in tension from point $\vec{R}$, the value of stress at which the material yields again is \underline{\hspace{1cm}} $MPa$
        \begin{figure}[!ht]
            \centering
            \resizebox{0.7\textwidth}{!}{%
\begin{circuitikz}
\tikzstyle{every node}=[font=\Large]
\draw [ line width=1pt ] (12.5,15) circle (0.25cm);
\draw [ line width=1pt ] (12.5,14.5) circle (0.25cm);
\draw [ line width=1pt ] (12.5,14) circle (0.25cm);
\draw [line width=1pt, short] (12.75,13.75) -- (12.75,15.25);
\draw [line width=1pt, short] (12.75,15.25) -- (13.75,15.25);
\draw [line width=1pt, short] (13.75,15.25) -- (14,15);
\draw [line width=1pt, short] (14,15) -- (14,14);
\draw [line width=1pt, short] (14,14) -- (13.75,13.75);
\draw [line width=1pt, short] (13.75,13.75) -- (12.75,13.75);
\draw [ line width=1pt ] (13.25,14.5) circle (0.25cm);
\draw [ line width=1pt ] (12.25,12.5) rectangle (11.25,16.25);
\draw [ line width=1pt ] (14,15) rectangle (27.5,14);
\draw [ line width=1pt ] (16.25,14.5) circle (0.25cm);
\draw [ line width=1pt ] (26.75,14.5) circle (0.25cm);
\draw [line width=1pt, short] (16,14) -- (16,8.75);
\draw [line width=1pt, short] (16.5,14) -- (16.5,8.75);
\draw [line width=1pt, short] (26.5,14) -- (26.5,7.5);
\draw [line width=1pt, short] (27,14) -- (27,7.5);
\draw [line width=1pt, short] (26.25,7.5) -- (27.25,7.5);
\draw [line width=1pt, short] (26.25,7.5) -- (26,7.25);
\draw [line width=1pt, short] (27.25,7.5) -- (27.5,7.25);
\draw [line width=1pt, short] (26,7.25) -- (26,6.25);
\draw [ line width=1pt ] (26.75,6.75) circle (0.25cm);
\draw [line width=1pt, short] (26,6.25) -- (27.5,6.25);
\draw [line width=1pt, short] (27.5,6.25) -- (27.5,7.25);
\draw [ line width=0.6pt ] (25,6.25) rectangle (28.75,5.25);
\draw [ line width=0.6pt ] (15,8.75) rectangle (17.5,7.75);
\draw [line width=0.6pt, dashed] (13.75,8.75) -- (15,8.75);
\draw [line width=0.6pt, dashed] (26.75,6.75) -- (30,6.75);
\draw [line width=0.6pt, dashed] (26.75,14.5) -- (30,14.5);
\draw [line width=0.6pt, dashed] (26.75,13.75) -- (26.75,12.5);
\draw [line width=0.6pt, dashed] (16.25,13.75) -- (16.25,12.5);
\draw [line width=0.6pt, <->, >=Stealth] (16.25,13) -- (26.75,13)node[pos=0.5,above, fill=white]{100cm};
\draw [line width=0.6pt, dashed] (16.25,16.25) -- (16.25,14.75);
\draw [line width=1pt, ->, >=Stealth] (21.25,17.5) -- (21.25,15)node[pos=0.5,right, fill=white]{P};
\draw [line width=1pt, <->, >=Stealth] (16.25,15.75) -- (21.25,15.75)node[pos=0.5,above, fill=white]{$a$};
\draw [line width=1pt, <->, >=Stealth] (28.75,14.5) -- (28.75,6.75)node[pos=0.5,right, fill=white]{125cm};
\draw [line width=1pt, dashed] (14,14.5) -- (16.25,14.5);
\draw [line width=1pt, <->, >=Stealth] (14.5,14.5) -- (14.5,8.75)node[pos=0.5,left, fill=white]{75cm};
\node [font=\Large] at (13.25,16) {$A$};
\node [font=\Large] at (16,15.5) {$B$};
\node [font=\Large] at (15.5,9.25) {$D$};
\node [font=\Large] at (26.5,15.5) {$C$};
\node [font=\Large] at (25.75,8) {$E$};
\end{circuitikz}
}

        \end{figure}

\subsection*{Q.26 to Q.55 carry one mark each.}

    %26
    \item The set of equations
        $$x+y+z=1$$
        $$ax-ay+3z=5$$
        $$5x-3y+az=6$$
        has infinite solutions if $a=$
        \begin{enumerate}
            \item $-3$
            \item $3$
            \item $4$
            \item $-4$
        \end{enumerate}

    %27
    \item A harmonic function is analytic if it satisfies the Laplace equation.

        If $u(x,y)=2x^2-2y^2+4xy$ is a harmonic function, then its conjugate harmonic function $v(x,y)$ is
        \begin{enumerate}
            \item $4xy-2x^2+2y^2+$ constant
            \item $4y^2-4xy+$ constant
            \item $2x^2-2y^2+xy+$ constant
            \item $-4xy+2y^2-2x^2+$ constant
        \end{enumerate}

    %28
    \item The variable $x$ takes a value between $0$ and $10$ with uniform probability distribution. The variable $y$ takes a value between $0$ and $20$ with uniform probability distribution. The probability of the sum of variables $(x + y)$ being greater than $20$ is
        \begin{enumerate}
            \item $0$
            \item $0.25$
            \item $0.33$
            \item $0.50$
        \end{enumerate}

    %29
    \item A car having weight $W$ is moving in the direction as shown in the figure. The center of gravity ($CG$) of the car is located at height $h$ from the ground, midway between the front and rear wheels. The distance between the front and rear wheels is $l$. The acceleration of the car is $a$, and acceleration due to gravity is $g$. The reactions on the front wheels $(R_{f})$ and rear wheels $(R_{r})$ are given by
        \begin{figure}[!ht]
            \centering
            \resizebox{0.6\textwidth}{!}{%
\begin{circuitikz}
\tikzstyle{every node}=[font=\Large]
\draw [ line width=1pt ] (12.5,16.25) rectangle (25,8.75);
\draw [line width=1pt, short] (17.5,16.25) -- (17.5,8.75);
\node [font=\Large] at (17.25,16.75) {$A$};
\node [font=\Large] at (25.25,16.75) {$B$};
\node [font=\Large] at (25.25,8.25) {$C$};
\node [font=\Large] at (17.25,8.25) {$D$};
\node [font=\Large] at (12.25,8.25) {$E$};
\node [font=\Large] at (12.25,16.75) {$F$};
\draw [line width=0.6pt, dashed] (12.5,8.75) -- (12.5,6.75);
\draw [line width=0.6pt, dashed] (17.5,8.75) -- (17.5,7);
\draw [line width=0.6pt, dashed] (25,8.75) -- (25,7.25);
\draw [line width=0.6pt, dashed] (25,8.75) -- (26.5,8.75);
\draw [line width=0.6pt, dashed] (25,16.25) -- (26.5,16.25);
\draw [line width=0.6pt, ->, >=Stealth] (20,17.5) .. controls (19,18.75) and (17.5,18.75) .. (16.25,17.5) ;
\node [font=\Large] at (18,19) {$12kNm$};
\draw [line width=0.6pt, <->, >=Stealth] (26,16.25) -- (26,8.75)node[pos=0.5,right, fill=white]{300mm};
\draw [line width=0.6pt, <->, >=Stealth] (17.5,7.75) -- (25,7.75)node[pos=0.5,below, fill=white]{300mm};
\draw [line width=0.6pt, <->, >=Stealth] (17.5,7.75) -- (12.5,7.75)node[pos=0.5,below, fill=white]{200mm};
\end{circuitikz}
}

        \end{figure}
        \begin{enumerate}
            \item $R_{f}=R_{r}=\frac{W}{2}-\frac{W}{g}\brak{\frac{h}{l}}a$
            \item $R_{f}=\frac{W}{2}+\frac{W}{g}\brak{\frac{h}{l}}a; R_{r}=\frac{W}{2}-\frac{W}{g}\brak{\frac{h}{l}}a$
            \item $R_{f}=\frac{W}{2}-\frac{W}{g}\brak{\frac{h}{l}}a; R_{r}=\frac{W}{2}+\frac{W}{g}\brak{\frac{h}{l}}a$
            \item $R_{f}=R_{r}=\frac{W}{2}+\frac{W}{g}\brak{\frac{h}{l}}a$
        \end{enumerate}

    \end{enumerate}

\end{document}
