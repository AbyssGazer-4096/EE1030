\let\negmedspace\undefined
\let\negthickspace\undefined
\documentclass[journal]{IEEEtran}
\usepackage[a5paper, margin=10mm, onecolumn]{geometry}
%\usepackage{lmodern} % Ensure lmodern is loaded for pdflatex
\usepackage{tfrupee} % Include tfrupee package

\setlength{\headheight}{1cm} % Set the height of the header box
\setlength{\headsep}{0mm}  % Set the distance between the header box and the top of the text

\usepackage{gvv-book}
\usepackage{gvv}
\usepackage{cite}
\usepackage{amsmath,amssymb,amsfonts,amsthm}
\usepackage{algorithmic}
\usepackage{graphicx}
\usepackage{textcomp}
\usepackage{xcolor}
\usepackage{txfonts}
\usepackage{listings}
\usepackage{enumitem}
\usepackage{mathtools}
\usepackage{gensymb}
\usepackage{comment}
\usepackage[breaklinks=true]{hyperref}
\usepackage{tkz-euclide} 
\usepackage{listings}
% \usepackage{gvv}                                        
\def\inputGnumericTable{}                                 
\usepackage[latin1]{inputenc}                                
\usepackage{color}
\usepackage{array}                                            
\usepackage{longtable}                                       
\usepackage{calc}                                             
\usepackage{multirow}                                         
\usepackage{hhline}                                           
\usepackage{ifthen}                                           
\usepackage{lscape}
\usepackage{tikz}
\usetikzlibrary{patterns}
\begin{document}

\bibliographystyle{IEEEtran}
\vspace{3cm}

\title{\textbf{ASSIGNMENT-3\\GATE XE-2014}}
\author{EE24BTECH11019 - DWARAK A}
\maketitle

\bigskip

\renewcommand{\thefigure}{\theenumi}
\renewcommand{\thetable}{\theenumi}

\section*{A : ENGINEERING MATHEMATICS}

\bigskip

\begin{enumerate}

\subsection*{Q.1 to Q.7 carry one mark each.}

    %14
    \item Ten chocolates are distributed randomly among three children standing in a row. The probability that the first child receives exactly three chocolates is 
        \begin{enumerate}
            \item $\frac{5\times2^{11}}{3^9}$
            \item $\frac{5\times2^{10}}{3^9}$
            \item $\frac{1}{3^9}$
            \item $\frac{1}{3}$
        \end{enumerate}
    
    %15
    \item Let the function $f:[5,0]\to\mathbb{R}$ be defined by

        $$f(x)=
        \begin{cases}
            2x + 5, & 0 \leq x < 1 \\ 
            2x^2 + 5, & 1 \leq x < 2 \\ 
            \frac{2}{3}x^3 + \frac{23}{3}, & 2 \leq x \leq 5 
        \end{cases}$$
        The number of points where $f$ is not differentiable in $(0,5)$, is \underline{\hspace{1.5cm}}.
    
    %16
    \item An integrating factor of the differential equation $\brak{3x^2y^3e^y+y^3+y^2}dx+\brak{x^3y^3e^y-xy}dy=0$ is
        \begin{enumerate}
            \item $\frac{1}{y}$
            \item $\frac{1}{y^2}$
            \item $\frac{1}{y^3}$
            \item $\ln{y}$
        \end{enumerate}

    %17
    \item If a cubic polynomial passes through the points $(0, 1)$, $(1, 0)$, $(2, 1)$ and $(3, 10)$, then it also passes through the point 
        \begin{enumerate}
            \item $(-2,-11)$
            \item $(-1,-2)$
            \item $(-1,-4)$
            \item $(-2,-23)$
        \end{enumerate}

\subsection*{Q.8 to Q.11 carry two marks each.}

    %18
    \item Let the function $f:[0,\infty)\to\mathbb{R}$ be such that $f^\prime(x)=\frac{8}{x^2+3x+4}$ for $x>0$ and $f(0)=1$. Then $f(1)$ lies in the interval
        \begin{enumerate}
            \item $[0,1]$
            \item $[2,3]$
            \item $[4,5]$
            \item $[6,7]$
        \end{enumerate}

    %19
    \item The perimeter of a rectangle having the largest area that can be inscribed in the ellipse $\frac{x^2}{8}+\frac{y^2}{32}=1$, is \underline{\hspace{1.5cm}}.

    %20
    \item If the work done in moving a particle once around a circle $x^2+y^2=4$ under the force field $\vec{F}(x,y)=(2x-ay)\hat{i}+(2y+ax)\hat{j}$ is $16\pi$, then $\abs{a}$ is equal to \underline{\hspace{1.5cm}}.

    %21
    \item Let $r$ and $s$ be real numbers. If $A = \myvec{1&2&0\\2&0&3\\r&s&0}$ and $b=\myvec{1\\1\\s-1}$, then the system of linear equations $AX=b$ has
        \begin{enumerate}
            \item no solutions for $s\neq2r$.
            \item infinitely many solutions for $s=2r\neq2$.
            \item a unique solution for $s=2r=2$.
            \item infinitely many solutions for $s=2r=2$.
        \end{enumerate}

\section*{B : FLUID MECHANICS}

\subsection*{Q.1 to Q.9 carry one mark each.}

    %22
    \item A dam with a curved shape is shown in the figure. The cross sectional area of the dam (shaded portion) is $100m^2$ and its centroid is at $\bar{x}=10m$. The vertical component of the hydrostatic force, $F_z$, is acting at a distance $x_p$. The value of $x_p$ is \underline{\hspace{1.5cm}}$m$.
        \begin{figure}[!ht]
            \centering
            \resizebox{0.35\textwidth}{!}{
\begin{circuitikz}
\tikzstyle{every node}=[font=\LARGE]
\draw  (7.5,15) rectangle (8.5,8.75);
\draw [ line width=0.6pt ] (8.5,13.75) rectangle (10.75,10.25);
\draw [line width=0.6pt, short] (7.5,12.5) -- (8.5,12.5);
\draw [line width=0.6pt, short] (8.5,12.5) -- (8.5,11);
\draw [line width=0.6pt, short] (8.5,11) -- (7.5,12.5);
\draw [line width=0.6pt, ->, >=Stealth] (8,12.5) -- (8,11.75);
\draw [line width=0.6pt, ->, >=Stealth] (8.25,12.5) -- (8.25,11.5);
\draw [line width=0.6pt, short] (7.5,15) -- (7,14.5);
\draw [line width=0.6pt, short] (7.5,14.5) -- (7,14);
\draw [line width=0.6pt, short] (7.5,14) -- (7,13.5);
\draw [line width=0.6pt, short] (7.5,13.5) -- (7,13);
\draw [line width=0.6pt, short] (7.5,13) -- (7,12.5);
\draw [line width=0.6pt, short] (7.5,12.5) -- (7,12);
\draw [line width=0.6pt, short] (7.5,12) -- (7,11.5);
\draw [line width=0.6pt, short] (7.5,11.5) -- (7,11);
\draw [line width=0.6pt, short] (7.5,11) -- (7,10.5);
\draw [line width=0.6pt, short] (7.5,10.5) -- (7,10);
\draw [line width=0.6pt, short] (7.5,9.5) -- (7,9);
\draw [line width=0.6pt, short] (7.5,9) -- (7,8.5);
\draw [line width=0.6pt, short] (7.5,16.25) -- (7.5,15.25);
\draw [line width=0.6pt, short] (8.5,16.25) -- (8.5,15.25);
\draw [line width=0.6pt, ->, >=Stealth] (6.5,15.75) -- (7.5,15.75);
\draw [line width=0.6pt, ->, >=Stealth] (9.5,15.75) -- (8.5,15.75);
\draw [line width=0.6pt, ->, >=Stealth] (5.75,7.5) -- (7.5,11.25);
\node at (5.75,7) {Impenetrable wall};
\node at (10,9.5) {$A=0.04m^2$};
\node [font = \large] at (9.5,13) {$m = 2.0 kg$};
\draw [line width=0.6pt, ->, >=Stealth] (9.5,12.25) -- (9.5,11)node[pos=0.5,left]{V};
\node at (11,15.75) {$h = 0.15 mm$};
\end{circuitikz}
}

        \end{figure}

    %23
    \item For an unsteady incompressible fluid flow, the velocity field is $\vec{V}=\brak{3x^2+3}t\,\hat{i}-6xyt\,\hat{j}$, where $x,y$ are in meters and $t$ is in seconds. Acceleration in $m/s^2$ at the point $x=10m$ and $y=0$, as measured by a stationary observer is
        \begin{enumerate}
            \item $303$
            \item $162$
            \item $43$
            \item $13$
        \end{enumerate}

    %24
    \item For an incompressible flow, the existence of components of acceleration for different types of flow is described in the table below.

        \begin{tabular}{l l}
            Type of Flow & Components of Acceleration \\
            P: Steady and uniform & 1: Local exists, convective does not exist \\
            Q: Steady and non-uniform & 2: Both exist \\
            R: Unsteady and uniform & 3: Both do not exist \\
            S: Unsteady and non-uniform & 4: Local does not exist, convective exists \\
        \end{tabular}

        Which one of the following options connecting the left column with the right column is correct?
        \begin{enumerate}
            \item P-$1$; Q-$4$; R-$3$; S-$2$
            \item P-$4$; Q-$1$; R-$2$; S-$3$
            \item P-$3$; Q-$2$; R-$1$; S-$4$
            \item P-$3$; Q-$4$; R-$1$; S-$2$
        \end{enumerate}

    %25
    \item Velocity in a two-dimensional flow field is specified as $u=x^2y;v=-y^2x$. The magnitude of the rate of angular deformation at a location $(x=2m \text{ and } y=1m)$ is \underline{\hspace{1cm}}$s^{-1}$.

    %26
    \item For a plane irrotational flow, equi-potential lines and streamlines are
        \begin{enumerate}
            \item parallel to each other.
            \item at an angle of $90\degree$ to each other.
            \item at an angle of $45\degree$ to each other.
            \item at an angle of $60\degree$ to each other.
        \end{enumerate}

    \end{enumerate}

\end{document}
