\let\negmedspace\undefined
\let\negthickspace\undefined
\documentclass[journal]{IEEEtran}
\usepackage[a5paper, margin=10mm, onecolumn]{geometry}
%\usepackage{lmodern} % Ensure lmodern is loaded for pdflatex
\usepackage{tfrupee} % Include tfrupee package

\setlength{\headheight}{1cm} % Set the height of the header box
\setlength{\headsep}{0mm}  % Set the distance between the header box and the top of the text

\usepackage{gvv-book}
\usepackage{gvv}
\usepackage{cite}
\usepackage{amsmath,amssymb,amsfonts,amsthm}
\usepackage{algorithmic}
\usepackage{graphicx}
\usepackage{textcomp}
\usepackage{xcolor}
\usepackage{txfonts}
\usepackage{listings}
\usepackage{enumitem}
\usepackage{mathtools}
\usepackage{gensymb}
\usepackage{comment}
\usepackage[breaklinks=true]{hyperref}
\usepackage{tkz-euclide} 
\usepackage{listings}
% \usepackage{gvv}                                        
\def\inputGnumericTable{}                                 
\usepackage[latin1]{inputenc}                                
\usepackage{color}
\usepackage{array}                                            
\usepackage{longtable}                                       
\usepackage{calc}                                             
\usepackage{multirow}                                         
\usepackage{hhline}                                           
\usepackage{ifthen}                                           
\usepackage{lscape}
\usepackage{tikz}
\usetikzlibrary{patterns}
\begin{document}

\bibliographystyle{IEEEtran}
\vspace{3cm}

\title{\textbf{ASSIGNMENT-2\\GATE XE-2008}}
\author{EE24BTECH11019 - DWARAK A}
\maketitle

\bigskip

\renewcommand{\thefigure}{\theenumi}
\renewcommand{\thetable}{\theenumi}

\section*{C : ELECTRICAL SCIENCES}

\bigskip

\begin{enumerate}

\subsection*{Q.9 to Q.30 carry two marks each.}

    %17
    \item In an induction motor the phase-difference, $\phi$, between the voltage applied at the stator terminals and the magnetizing current is
        \begin{enumerate}
            \item $\phi=0^\degree$
            \item $0^\degree<\phi<90^\degree$
            \item $\phi=90^\degree$
            \item $90^\degree<\phi<180^\degree$
        \end{enumerate}
    
    %18
    \item A voltage of $+5V$ is applied (with respect to ground) to both the inputs $V_1$ and $V_2$ of an operational amplifier circuit shown in the figure. $R_1=20k\ohm$ and $R_2=10k\ohm$. The output voltage, $V_o$ is
        \begin{figure}[!ht]
            \centering
            \resizebox{0.35\textwidth}{!}{
\begin{circuitikz}
\tikzstyle{every node}=[font=\LARGE]
\draw  (7.5,15) rectangle (8.5,8.75);
\draw [ line width=0.6pt ] (8.5,13.75) rectangle (10.75,10.25);
\draw [line width=0.6pt, short] (7.5,12.5) -- (8.5,12.5);
\draw [line width=0.6pt, short] (8.5,12.5) -- (8.5,11);
\draw [line width=0.6pt, short] (8.5,11) -- (7.5,12.5);
\draw [line width=0.6pt, ->, >=Stealth] (8,12.5) -- (8,11.75);
\draw [line width=0.6pt, ->, >=Stealth] (8.25,12.5) -- (8.25,11.5);
\draw [line width=0.6pt, short] (7.5,15) -- (7,14.5);
\draw [line width=0.6pt, short] (7.5,14.5) -- (7,14);
\draw [line width=0.6pt, short] (7.5,14) -- (7,13.5);
\draw [line width=0.6pt, short] (7.5,13.5) -- (7,13);
\draw [line width=0.6pt, short] (7.5,13) -- (7,12.5);
\draw [line width=0.6pt, short] (7.5,12.5) -- (7,12);
\draw [line width=0.6pt, short] (7.5,12) -- (7,11.5);
\draw [line width=0.6pt, short] (7.5,11.5) -- (7,11);
\draw [line width=0.6pt, short] (7.5,11) -- (7,10.5);
\draw [line width=0.6pt, short] (7.5,10.5) -- (7,10);
\draw [line width=0.6pt, short] (7.5,9.5) -- (7,9);
\draw [line width=0.6pt, short] (7.5,9) -- (7,8.5);
\draw [line width=0.6pt, short] (7.5,16.25) -- (7.5,15.25);
\draw [line width=0.6pt, short] (8.5,16.25) -- (8.5,15.25);
\draw [line width=0.6pt, ->, >=Stealth] (6.5,15.75) -- (7.5,15.75);
\draw [line width=0.6pt, ->, >=Stealth] (9.5,15.75) -- (8.5,15.75);
\draw [line width=0.6pt, ->, >=Stealth] (5.75,7.5) -- (7.5,11.25);
\node at (5.75,7) {Impenetrable wall};
\node at (10,9.5) {$A=0.04m^2$};
\node [font = \large] at (9.5,13) {$m = 2.0 kg$};
\draw [line width=0.6pt, ->, >=Stealth] (9.5,12.25) -- (9.5,11)node[pos=0.5,left]{V};
\node at (11,15.75) {$h = 0.15 mm$};
\end{circuitikz}
}

        \end{figure}
        \begin{enumerate}
            \item $-5V$
            \item $0V$
            \item $5V$
            \item $20V$
        \end{enumerate}
    
    %19
    \item A pair of zener diodes each with a forward drop of $0.7V$ and a zener voltage of $4.7V$ is connected as shown in the figure. The input voltage is $v_{in}=10sin(2t)$. The peak-to-peak output voltage, $v_o$, is
        \begin{figure}[!ht]
            \centering
            \resizebox{0.7\textwidth}{!}{%
\begin{circuitikz}
\tikzstyle{every node}=[font=\Large]
\draw [ line width=1pt ] (12.5,15) circle (0.25cm);
\draw [ line width=1pt ] (12.5,14.5) circle (0.25cm);
\draw [ line width=1pt ] (12.5,14) circle (0.25cm);
\draw [line width=1pt, short] (12.75,13.75) -- (12.75,15.25);
\draw [line width=1pt, short] (12.75,15.25) -- (13.75,15.25);
\draw [line width=1pt, short] (13.75,15.25) -- (14,15);
\draw [line width=1pt, short] (14,15) -- (14,14);
\draw [line width=1pt, short] (14,14) -- (13.75,13.75);
\draw [line width=1pt, short] (13.75,13.75) -- (12.75,13.75);
\draw [ line width=1pt ] (13.25,14.5) circle (0.25cm);
\draw [ line width=1pt ] (12.25,12.5) rectangle (11.25,16.25);
\draw [ line width=1pt ] (14,15) rectangle (27.5,14);
\draw [ line width=1pt ] (16.25,14.5) circle (0.25cm);
\draw [ line width=1pt ] (26.75,14.5) circle (0.25cm);
\draw [line width=1pt, short] (16,14) -- (16,8.75);
\draw [line width=1pt, short] (16.5,14) -- (16.5,8.75);
\draw [line width=1pt, short] (26.5,14) -- (26.5,7.5);
\draw [line width=1pt, short] (27,14) -- (27,7.5);
\draw [line width=1pt, short] (26.25,7.5) -- (27.25,7.5);
\draw [line width=1pt, short] (26.25,7.5) -- (26,7.25);
\draw [line width=1pt, short] (27.25,7.5) -- (27.5,7.25);
\draw [line width=1pt, short] (26,7.25) -- (26,6.25);
\draw [ line width=1pt ] (26.75,6.75) circle (0.25cm);
\draw [line width=1pt, short] (26,6.25) -- (27.5,6.25);
\draw [line width=1pt, short] (27.5,6.25) -- (27.5,7.25);
\draw [ line width=0.6pt ] (25,6.25) rectangle (28.75,5.25);
\draw [ line width=0.6pt ] (15,8.75) rectangle (17.5,7.75);
\draw [line width=0.6pt, dashed] (13.75,8.75) -- (15,8.75);
\draw [line width=0.6pt, dashed] (26.75,6.75) -- (30,6.75);
\draw [line width=0.6pt, dashed] (26.75,14.5) -- (30,14.5);
\draw [line width=0.6pt, dashed] (26.75,13.75) -- (26.75,12.5);
\draw [line width=0.6pt, dashed] (16.25,13.75) -- (16.25,12.5);
\draw [line width=0.6pt, <->, >=Stealth] (16.25,13) -- (26.75,13)node[pos=0.5,above, fill=white]{100cm};
\draw [line width=0.6pt, dashed] (16.25,16.25) -- (16.25,14.75);
\draw [line width=1pt, ->, >=Stealth] (21.25,17.5) -- (21.25,15)node[pos=0.5,right, fill=white]{P};
\draw [line width=1pt, <->, >=Stealth] (16.25,15.75) -- (21.25,15.75)node[pos=0.5,above, fill=white]{$a$};
\draw [line width=1pt, <->, >=Stealth] (28.75,14.5) -- (28.75,6.75)node[pos=0.5,right, fill=white]{125cm};
\draw [line width=1pt, dashed] (14,14.5) -- (16.25,14.5);
\draw [line width=1pt, <->, >=Stealth] (14.5,14.5) -- (14.5,8.75)node[pos=0.5,left, fill=white]{75cm};
\node [font=\Large] at (13.25,16) {$A$};
\node [font=\Large] at (16,15.5) {$B$};
\node [font=\Large] at (15.5,9.25) {$D$};
\node [font=\Large] at (26.5,15.5) {$C$};
\node [font=\Large] at (25.75,8) {$E$};
\end{circuitikz}
}

        \end{figure}
        \begin{enumerate}
            \item $5.4V$
            \item $4.7V$
            \item $1.4V$
            \item $0.7V$
        \end{enumerate}

    %20
    \item The npn transistor shown in figure has $h_{fe}=99$ and $V_{BE}=0.7V$. Under quiescent condition, $V_{EG}=4.3V$ and $I_{E}=1mA$, and the current in $R_2$ is $0.1 mA$. The value of $R_1$, required for biasing the circuit is
        \begin{figure}[!ht]
            \centering
            \resizebox{0.6\textwidth}{!}{%
\begin{circuitikz}
\tikzstyle{every node}=[font=\Large]
\draw [ line width=1pt ] (12.5,16.25) rectangle (25,8.75);
\draw [line width=1pt, short] (17.5,16.25) -- (17.5,8.75);
\node [font=\Large] at (17.25,16.75) {$A$};
\node [font=\Large] at (25.25,16.75) {$B$};
\node [font=\Large] at (25.25,8.25) {$C$};
\node [font=\Large] at (17.25,8.25) {$D$};
\node [font=\Large] at (12.25,8.25) {$E$};
\node [font=\Large] at (12.25,16.75) {$F$};
\draw [line width=0.6pt, dashed] (12.5,8.75) -- (12.5,6.75);
\draw [line width=0.6pt, dashed] (17.5,8.75) -- (17.5,7);
\draw [line width=0.6pt, dashed] (25,8.75) -- (25,7.25);
\draw [line width=0.6pt, dashed] (25,8.75) -- (26.5,8.75);
\draw [line width=0.6pt, dashed] (25,16.25) -- (26.5,16.25);
\draw [line width=0.6pt, ->, >=Stealth] (20,17.5) .. controls (19,18.75) and (17.5,18.75) .. (16.25,17.5) ;
\node [font=\Large] at (18,19) {$12kNm$};
\draw [line width=0.6pt, <->, >=Stealth] (26,16.25) -- (26,8.75)node[pos=0.5,right, fill=white]{300mm};
\draw [line width=0.6pt, <->, >=Stealth] (17.5,7.75) -- (25,7.75)node[pos=0.5,below, fill=white]{300mm};
\draw [line width=0.6pt, <->, >=Stealth] (17.5,7.75) -- (12.5,7.75)node[pos=0.5,below, fill=white]{200mm};
\end{circuitikz}
}

        \end{figure}
        \begin{enumerate}
            \item $10.1k\ohm$
            \item $90.9k\ohm$
            \item $100.1k\ohm$
            \item $150.2k\ohm$
        \end{enumerate}

    %21
    \item The forward characteristics of a p-n diode is given by $i=I_{s}e^{\frac{v}{\brak{nV_{T}}}}$ with $n=2$ and $V_{T}=25mV$. If the diode current is measured to be $100mA$ at $0.7V$ drop, the diode power dissipation at a diode current of $200mA$ is 
        \begin{enumerate}
            \item $70mW$
            \item $140mW$
            \item $143mW$
            \item $147mW$
        \end{enumerate}

    %22
    \item For the n-channel JFET shown in the figure the pinch-off voltage, $V_{p}=-5V$, and gate source voltage, $V_{GS}=-3V$. The minimum required drain to source voltage, $V_{DS}$ to operate at pinch-off condition is
        \begin{figure}[!ht]
            \centering
            \resizebox{0.4\textwidth}{!}{%
\begin{circuitikz}
\tikzstyle{every node}=[font=\LARGE]
\draw [line width=1pt, ->, >=Stealth] (5,7.5) -- (5,16.25);
\draw [line width=1pt, ->, >=Stealth] (5,7.5) -- (15,7.5);
\draw [line width=1pt, dashed] (5,15) -- (13,15);
\draw [line width=1pt, dashed] (13,15) -- (13,7.5);
\draw [line width=1pt, dashed] (5,10) -- (12.75,10);
\draw [line width=1pt, dashed] (11.25,10) -- (11.25,7.75);
\draw [line width=1pt, short] (13,15) -- (11.25,10);
\draw [line width=1pt, short] (5,7.5) .. controls (7.75,14.25) and (7.5,13.75) .. (14.5,15.25);
\draw [line width=1pt, short] (8.25,13.75) -- (8.75,13.75);
\draw [line width=1pt, short] (8.75,13.75) -- (8.5,13.25);
\draw [line width=1pt, short] (11.5,11.5) -- (11.5,11);
\draw [line width=1pt, short] (11.5,11) -- (12,11.25);
\node [font=\LARGE] at (11,10.5) {B};
\node [font=\LARGE] at (13,15.5) {A};
\node [font=\LARGE] at (4.25,10) {$100$};
\node [font=\LARGE] at (4.25,15) {$500$};
\node [font=\LARGE] at (5,17.25) {$\sigma$};
\node [font=\LARGE] at (5,16.75) {$(MPa)$};
\node [font=\LARGE] at (13,7) {$0.5$};
\node [font=\LARGE] at (15,7) {$\varepsilon$};
\draw [line width=1pt, short] (5,7) -- (5,6.25);
\draw [line width=1pt, short] (11.25,7) -- (11.25,6.25);
\draw [line width=1pt, <->, >=Stealth] (5,6.75) -- (11.25,6.75)node[pos=0.5, fill=white]{$\varepsilon_{B}$};
\end{circuitikz}
}

        \end{figure}
        \begin{enumerate}
            \item $0V$
            \item $2V$
            \item $5V$
            \item $8V$
        \end{enumerate}

    %23
    \item The Boolean function corresponding to the truth table shown is
        \begin{table}[!ht]
            \centering
            \begin{tabular}{|l|c|}
\hline
Month    & Demand \\
\hline
January  & 450    \\
\hline
February & 440    \\
\hline
March    & 460    \\
\hline
April    & 510    \\
\hline
May      & 520    \\
\hline
June     & 495    \\
\hline
July     & 475    \\
\hline
August   & 560    \\
\hline

\end{tabular}

        \end{table}
        \begin{enumerate}
            \item $F=A\overline{B}C+\overline{A}BC+\overline{AB}C+\overline{ABC}$
            \item $F=ABC+AB\overline{C}+\overline{A}BC$
            \item $F=ABC+AB\overline{C}+A\overline{BC}+\overline{A}B\overline{C}$
            \item $F=A\overline{B}C+\overline{A}BC+\overline{AB}C+\overline{A}BC$
        \end{enumerate}

    %24
    \item The decimal number $328$ when converted to the base of $9$ is equivalent to 
        \begin{enumerate}
            \item $(434)_9$
            \item $(424)_9$
            \item $(404)_9$
            \item $(304)_9$
        \end{enumerate}

    %25
    \item The following logic circuit can be represented by the Boolean expression
        \begin{figure}[!ht]
            \centering
            \resizebox{0.6\textwidth}{!}{%
\begin{circuitikz}
\tikzstyle{every node}=[font=\normalsize]
\draw (8,11.25) to[short] (8.25,11.25);
\draw (8,10.75) to[short] (8.25,10.75);
\draw (8.25,11.25) node[ieeestd nor port, anchor=in 1, scale=0.89](port){} (port.out) to[short] (10,11);
\draw (8,8.75) to[short] (8.25,8.75);
\draw (8,8.25) to[short] (8.25,8.25);
\draw (8.25,8.75) node[ieeestd and port, anchor=in 1, scale=0.89](port){} (port.out) to[short] (10,8.5);
\draw (11,11) to[short] (11.25,11);
\draw (11,10.5) to[short] (11.25,10.5);
\draw (11.25,11) node[ieeestd xnor port, anchor=in 1, scale=0.89](port){} (port.out) to[short] (13,10.75);
\draw (11.25,8.5) node[ieeestd not port, anchor=in](port){} (port.out) to[short] (13,8.5);
\draw (port.in) to[short] (11,8.5);
\draw (14.5,10) to[short] (14.75,10);
\draw (14.5,9.5) to[short] (14.75,9.5);
\draw (14.75,10) node[ieeestd or port, anchor=in 1, scale=0.89](port){} (port.out) to[short] (16.5,9.75);
\draw [ line width=0.2pt](10,11) to[short] (11,11);
\draw [ line width=0.2pt](13,10.75) to[short] (13.75,10.75);
\draw [ line width=0.2pt](13.75,10.75) to[short] (13.75,10);
\draw [ line width=0.2pt](13.75,10) to[short] (14.5,10);
\draw [ line width=0.2pt](13,8.5) to[short] (13.75,8.5);
\draw [ line width=0.2pt](13.75,8.5) to[short] (13.75,9.5);
\draw [ line width=0.2pt](13.75,9.5) to[short] (14.5,9.5);
\draw [ line width=0.2pt](10,8.5) to[short] (11,8.5);
\draw [ line width=0.2pt](11,10.5) to[short] (10.5,10.5);
\draw [ line width=0.2pt](10.5,10.5) to[short] (10.5,8.5);
\draw [ line width=0.2pt](8,11.25) to[short] (6.25,11.25);
\draw [ line width=0.2pt](8,8.25) to[short] (6.25,8.25);
\draw [ line width=0.2pt](8,8.75) to[short] (7.5,8.75);
\draw [ line width=0.2pt](8,10.75) to[short] (6.25,10.75);
\draw [ line width=0.2pt](7.5,10.75) to[short] (7.5,8.75);
\node [font=\normalsize] at (17,9.75) {$F$};
\node [font=\normalsize] at (6,11.25) {$A$};
\node [font=\normalsize] at (6,10.75) {$B$};
\node [font=\normalsize] at (6,8.25) {$C$};
\end{circuitikz}
}

        \end{figure}
        \begin{enumerate}
            \item $F=\overline{B}+BC+\overline{C}$
            \item $F=\overline{B}+\overline{C}$
            \item $F=\brak{\overline{B+C}}$
            \item $F=\overline{A}+\overline{B}+\overline{C}$
        \end{enumerate}

    %26
    \item A $4$-bit resistor network based D/A converter is shown in the figure. The output corresponding to the number $1010$ is
        \begin{figure}[!ht]
            \centering
            \resizebox{0.3\textwidth}{!}{%
\begin{circuitikz}
\tikzstyle{every node}=[font=\Huge]
\draw [line width=0.8pt, short] (1.25,16.25) -- (8.75,10);
\draw [line width=0.8pt, short] (1.25,13.75) -- (8.75,7.5);
\draw [line width=0.8pt, short] (8.75,10) .. controls (10.5,8.5) and (12.25,8.25) .. (13.75,10);
\draw [line width=0.8pt, short] (8.75,7.5) .. controls (10.5,6) and (12.25,5.75) .. (13.75,7.5);
\draw [line width=0.8pt, short] (3,15.25) -- (5,13.5);
\draw [line width=0.8pt, short] (5,13.5) -- (6.25,14.75);
\draw [line width=0.8pt, short] (6.25,14.75) -- (4.25,16.5);
\draw [line width=0.8pt, short] (4.25,16.5) -- (3,15.25);
\draw [line width=0.8pt, dashed] (11.25,16.25) -- (11.25,6.25);
\node at (2.75,16.5) {P};
\node at (11.25,5.5) {Q};
\end{circuitikz}
}

        \end{figure}
        \begin{enumerate}
            \item $5.0V$
            \item $6.25V$
            \item $7.25V$
            \item $10.0V$
        \end{enumerate}

    %27
    \item Two $10V$ square waves of same frequency but $90^\degree$ out-of-phase to each other are applied to $X$ and $Y$ deflecting plates of a CRO. Both channels are set at $5V$/division and the CRO is operating in the $X-Y$ mode. The display on CRO will be 
        \begin{enumerate}
            \item A bright circle
            \item A bright ellipse
            \item Two bright spots at the diagonal of a faint square
            \item Four bright spots at the comers of a faint square
        \end{enumerate}

    %28
    \item CRO that is used in $X-Y$ mode displays a line inclined at an angle of $135^\degree$. The $X$-channel gain is $5V$/division and the $Y$-channel gain is $10V$/division. If the display point at a given instant corresponds to $+3$ divisions on the $X$-axis, the input voltage to the $Y$-channel at that instant is 
        \begin{enumerate}
            \item $-30V$
            \item $-15V$
            \item $+15V$
            \item $+30V$
        \end{enumerate}

\subsection*{Common Data Questions}

Common Data for Questions 29 and 30:

A $1.0kW$ induction motor has $15$ pole-pairs and is supplied from a $60Hz$ source. The motor runs at $0.05$ slip. The stator loss is $80W$.
    
    %29
    \item The speed of the rotating magnetic field in the motor and the frequency of the rotor induced voltage are
        \begin{enumerate}
            \item $120rpm, 1.5Hz$
            \item $120rpm, 28.5Hz$
            \item $240rpm, 3.0Hz$
            \item $240rpm, 57.0Hz$
        \end{enumerate}

    %30
    \item The rotor copper loss of this induction motor is
        \begin{enumerate}
            \item $120rpm, 1.5Hz$
            \item $120rpm, 28.5Hz$
            \item $240rpm, 3.0Hz$
            \item $240rpm, 57.0Hz$
        \end{enumerate}

\subsection*{Linked Answer Questions: Q.31 to Q.34 carry two marks each.}

Statement for Linked Answer Questions 31 and 32:

A practical dc voltage source is represented as an ideal dc voltage source in series with an internal resistance. The $V-I$ characteristics of two such sources, $E_1$ and $E_2$, are shown in the figure.
        \begin{figure}[!ht]
            \centering
            \resizebox{0.5\textwidth}{!}{%
\begin{circuitikz}
\tikzstyle{every node}=[font=\large]
\draw [->, >=Stealth] (8.75,8.75) -- (8.75,15.75);
\draw [->, >=Stealth] (8.75,8.75) -- (17.5,8.75);
\draw [dashed] (16.25,8.75) -- (16.25,13.75);
\draw [short] (8.75,13.75) -- (16.25,13.25)node[pos=0.5,below]{$E_2$};
\draw [short] (8.75,15) -- (16.25,13.75)node[pos=0.5,above]{$E_1$};
\node [font=\large] at (8,15) {$100V$};
\node [font=\large] at (8,13.5) {$80V$};
\node [font=\large] at (16.75,13.75) {$80V$};
\node [font=\large] at (16.75,13.25) {$72V$};
\node [font=\large] at (8.75,8.5) {$0A$};
\node [font=\large] at (12.5,8.25) {Current};
\node [font=\large] at (7.5,11.75) {Voltage};
\node [font=\large] at (16.25,8.5) {$4A$};
\draw[domain=8.25:9.25,samples=100,smooth] plot (\x,{0.15*sin(6.3*\x r -8.25 r ) +10});
\draw[domain=8.25:9.25,samples=100,smooth] plot (\x,{0.15*sin(6.3*\x r -8.25 r ) +9.75});
\end{circuitikz}
}

        \end{figure}

    %31
    \item The respective internal resistances of $E_1$ and $E_2$ are
        \begin{enumerate}
            \item $20\ohm,8\ohm$
            \item $5\ohm,2\ohm$
            \item $8\ohm,20\ohm$
            \item $2\ohm,5\ohm$
        \end{enumerate}

    %32
    \item If the two sources, $E_1$ and $E_2$, in question Q.31 are connected in parallel to feed a load of $200\ohm$ resistance, then the load current is in the range 
        \begin{enumerate}
            \item $0.0A$ to $0.5A$
            \item $0.5A$ to $2.0A$
            \item $2.0A$ to $4.0A$
            \item $4.0A$ to $8.0A$
        \end{enumerate}

Statement for Linked Answer Questions 33 and 34:

A function $F$, in "Sum of Product (SOP)" form is described by $$F=\sum m(0,1,3,4,5,6,7,13,15)$$

    %33
    \item The Karnaugh Map for $F$ is given by ($X$ being don't care)
        \begin{enumerate}
            \item $$
                \begin{array}{|c|c|c|c|c|}
                \hline
                    \begin{tikzpicture}
        \node at (0.1,0.1) {CD};
        \draw[thick] (0.4,-0.4) -- (-1.2,0.4);
        \node[anchor=south east] at (-0.5,-0.5) {AB};
    \end{tikzpicture}
                & 00 & 01 & 11 & 10 \\
                \hline
                00 & X & X & X & 1 \\
                \hline
                01 & X & X & X & X \\
                \hline
                11 & 1 & X & X & 1 \\
                \hline
                10 & 1 & 1 & 1 & 1 \\
                \hline
            \end{array}
            $$
            \item $$
                \begin{array}{|c|c|c|c|c|}
                \hline
                    \begin{tikzpicture}
        \node at (0.1,0.1) {CD};
        \draw[thick] (0.4,-0.4) -- (-1.2,0.4);
        \node[anchor=south east] at (-0.5,-0.5) {AB};
    \end{tikzpicture}
                & 00 & 01 & 11 & 10 \\
                \hline
                00 & 1 & 1 & 1 & X \\
                \hline
                01 & 1 & 1 & 1 & 1 \\
                \hline
                11 & X & 1 & 1 & x \\
                \hline
                10 & X & X & X & X \\
                \hline
            \end{array}
            $$
            \item $$
                \begin{array}{|c|c|c|c|c|}
                \hline
                    \begin{tikzpicture}
        \node at (0.1,0.1) {CD};
        \draw[thick] (0.4,-0.4) -- (-1.2,0.4);
        \node[anchor=south east] at (-0.5,-0.5) {AB};
    \end{tikzpicture}
                & 00 & 01 & 11 & 10 \\
                \hline
                00 & 1 & X & 1 & X \\
                \hline
                01 & X & 1 & X & 1 \\
                \hline
                11 & 1 & X & X & X \\
                \hline
                10 & X & 1 & X & 1 \\
                \hline
            \end{array}
            $$
            \item $$
                \begin{array}{|c|c|c|c|c|}
                \hline
                    \begin{tikzpicture}
        \node at (0.1,0.1) {CD};
        \draw[thick] (0.4,-0.4) -- (-1.2,0.4);
        \node[anchor=south east] at (-0.5,-0.5) {AB};
    \end{tikzpicture}
                & 00 & 01 & 11 & 10 \\
                \hline
                00 & 1 & 1 & X & X \\
                \hline
                01 & X & X & X & X \\
                \hline
                11 & X & 1 & 1 & 1 \\
                \hline
                10 & X & 1 & 1 & X \\
                \hline
            \end{array}
            $$
        \end{enumerate}

    %34
    \item Using the Karnaugh Map obtained in question Q.33, the function, $F$ reduces to 
        \begin{enumerate}
            \item $F=\overline{AC}+\overline{A}D+AB+BD$
            \item $F=AC+AD+\overline{AB}+\overline{BD}$
            \item $F=AC+\overline{A}D+\overline{AB}+\overline{BD}$
            \item $F=\overline{AC}+\overline{A}D+\overline{A}B+BD$
        \end{enumerate}

\end{enumerate}

\end{document}
