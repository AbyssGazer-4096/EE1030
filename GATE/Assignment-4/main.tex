\let\negmedspace\undefined
\let\negthickspace\undefined
\documentclass[journal]{IEEEtran}
\usepackage[a5paper, margin=10mm, onecolumn]{geometry}
%\usepackage{lmodern} % Ensure lmodern is loaded for pdflatex
\usepackage{tfrupee} % Include tfrupee package

\setlength{\headheight}{1cm} % Set the height of the header box
\setlength{\headsep}{0mm}  % Set the distance between the header box and the top of the text

\usepackage{gvv-book}
\usepackage{gvv}
\usepackage{cite}
\usepackage{amsmath,amssymb,amsfonts,amsthm}
\usepackage{algorithmic}
\usepackage{graphicx}
\usepackage{textcomp}
\usepackage{xcolor}
\usepackage{txfonts}
\usepackage{listings}
\usepackage{enumitem}
\usepackage{mathtools}
\usepackage{gensymb}
\usepackage{comment}
\usepackage[breaklinks=true]{hyperref}
\usepackage{tkz-euclide} 
\usepackage{listings}
% \usepackage{gvv}                                        
\def\inputGnumericTable{}                                 
\usepackage[latin1]{inputenc}                                
\usepackage{color}
\usepackage{array}                                            
\usepackage{longtable}                                       
\usepackage{calc}                                             
\usepackage{multirow}                                         
\usepackage{hhline}                                           
\usepackage{ifthen}                                           
\usepackage{lscape}
\usepackage{tikz}
\usetikzlibrary{patterns}
\begin{document}

\bibliographystyle{IEEEtran}
\vspace{3cm}

\title{\textbf{ASSIGNMENT-4\\GATE ME-2018}}
\author{EE24BTECH11019 - DWARAK A}
\maketitle

\bigskip

\renewcommand{\thefigure}{\theenumi}
\renewcommand{\thetable}{\theenumi}

\begin{enumerate}

\subsection*{Q.26 to Q.55 carry two marks each.}

    %43
    \item A solid block of $2.0 kg$ mass slides steadily at a velocity $V$ along a vertical wall as shown in the figure below. A thin oil film of thickness $h=0.15mm$ provides lubrication between the block and the wall. The surface area of the face of the block in contact with the oil film is $0.04 m^2$ . The velocity distribution within the oil film gap is linear as shown in the figure. Take dynamic viscosity of oil as $7\times10^{-3} Pa-s$ and acceleration due to gravity as $10 m/s^2$ . Neglect weight of the oil. The terminal velocity $V$ (in $m/s$) of the block is \underline{\hspace{1cm}} (correct to one decimal place).
        \begin{figure}[!ht]
            \centering
            \resizebox{0.35\textwidth}{!}{
\begin{circuitikz}
\tikzstyle{every node}=[font=\LARGE]
\draw  (7.5,15) rectangle (8.5,8.75);
\draw [ line width=0.6pt ] (8.5,13.75) rectangle (10.75,10.25);
\draw [line width=0.6pt, short] (7.5,12.5) -- (8.5,12.5);
\draw [line width=0.6pt, short] (8.5,12.5) -- (8.5,11);
\draw [line width=0.6pt, short] (8.5,11) -- (7.5,12.5);
\draw [line width=0.6pt, ->, >=Stealth] (8,12.5) -- (8,11.75);
\draw [line width=0.6pt, ->, >=Stealth] (8.25,12.5) -- (8.25,11.5);
\draw [line width=0.6pt, short] (7.5,15) -- (7,14.5);
\draw [line width=0.6pt, short] (7.5,14.5) -- (7,14);
\draw [line width=0.6pt, short] (7.5,14) -- (7,13.5);
\draw [line width=0.6pt, short] (7.5,13.5) -- (7,13);
\draw [line width=0.6pt, short] (7.5,13) -- (7,12.5);
\draw [line width=0.6pt, short] (7.5,12.5) -- (7,12);
\draw [line width=0.6pt, short] (7.5,12) -- (7,11.5);
\draw [line width=0.6pt, short] (7.5,11.5) -- (7,11);
\draw [line width=0.6pt, short] (7.5,11) -- (7,10.5);
\draw [line width=0.6pt, short] (7.5,10.5) -- (7,10);
\draw [line width=0.6pt, short] (7.5,9.5) -- (7,9);
\draw [line width=0.6pt, short] (7.5,9) -- (7,8.5);
\draw [line width=0.6pt, short] (7.5,16.25) -- (7.5,15.25);
\draw [line width=0.6pt, short] (8.5,16.25) -- (8.5,15.25);
\draw [line width=0.6pt, ->, >=Stealth] (6.5,15.75) -- (7.5,15.75);
\draw [line width=0.6pt, ->, >=Stealth] (9.5,15.75) -- (8.5,15.75);
\draw [line width=0.6pt, ->, >=Stealth] (5.75,7.5) -- (7.5,11.25);
\node at (5.75,7) {Impenetrable wall};
\node at (10,9.5) {$A=0.04m^2$};
\node [font = \large] at (9.5,13) {$m = 2.0 kg$};
\draw [line width=0.6pt, ->, >=Stealth] (9.5,12.25) -- (9.5,11)node[pos=0.5,left]{V};
\node at (11,15.75) {$h = 0.15 mm$};
\end{circuitikz}
}

        \end{figure}

    %44
    \item A tank of volume $0.05 m^3$ contains a mixture of saturated water and saturated steam at $200^\degree C$.The mass of the liquid present is $8 kg$. The entropy (in $kJ/kg K$) of the mixture is \underline{\hspace{1cm}} (correct to two decimal places).

        Property data for saturated steam and water are:

        At $200^\degree C$,$p_{sat}= 1.5538 MPa$

        $v_{f} = 0.001157 m^3/kg$,$v_{g} = 0.12736 m^3/kg$

        $s_{fg} = 4.1014 kJ/kg K$, $s_{f} = 2.3309 kJ/kg K$
    
    %45
    \item Steam flows through a nozzle at a mass flow rate of $\dot{m}=0.1kg/s$ with a heat loss of 5 kW. The enthalpies at inlet and exit are 2500 kJ/kg and 2350 kJ/kg, respectively. Assuming
        negligible velocity at inlet $(C_1\approx0)$, the velocity $C_2$ of steam (in $m/s$) at the nozzle exit is \underline{\hspace{1cm}} (correct to two decimal places).
        \begin{figure}[!ht]
            \centering
            \resizebox{0.7\textwidth}{!}{%
\begin{circuitikz}
\tikzstyle{every node}=[font=\Large]
\draw [ line width=1pt ] (12.5,15) circle (0.25cm);
\draw [ line width=1pt ] (12.5,14.5) circle (0.25cm);
\draw [ line width=1pt ] (12.5,14) circle (0.25cm);
\draw [line width=1pt, short] (12.75,13.75) -- (12.75,15.25);
\draw [line width=1pt, short] (12.75,15.25) -- (13.75,15.25);
\draw [line width=1pt, short] (13.75,15.25) -- (14,15);
\draw [line width=1pt, short] (14,15) -- (14,14);
\draw [line width=1pt, short] (14,14) -- (13.75,13.75);
\draw [line width=1pt, short] (13.75,13.75) -- (12.75,13.75);
\draw [ line width=1pt ] (13.25,14.5) circle (0.25cm);
\draw [ line width=1pt ] (12.25,12.5) rectangle (11.25,16.25);
\draw [ line width=1pt ] (14,15) rectangle (27.5,14);
\draw [ line width=1pt ] (16.25,14.5) circle (0.25cm);
\draw [ line width=1pt ] (26.75,14.5) circle (0.25cm);
\draw [line width=1pt, short] (16,14) -- (16,8.75);
\draw [line width=1pt, short] (16.5,14) -- (16.5,8.75);
\draw [line width=1pt, short] (26.5,14) -- (26.5,7.5);
\draw [line width=1pt, short] (27,14) -- (27,7.5);
\draw [line width=1pt, short] (26.25,7.5) -- (27.25,7.5);
\draw [line width=1pt, short] (26.25,7.5) -- (26,7.25);
\draw [line width=1pt, short] (27.25,7.5) -- (27.5,7.25);
\draw [line width=1pt, short] (26,7.25) -- (26,6.25);
\draw [ line width=1pt ] (26.75,6.75) circle (0.25cm);
\draw [line width=1pt, short] (26,6.25) -- (27.5,6.25);
\draw [line width=1pt, short] (27.5,6.25) -- (27.5,7.25);
\draw [ line width=0.6pt ] (25,6.25) rectangle (28.75,5.25);
\draw [ line width=0.6pt ] (15,8.75) rectangle (17.5,7.75);
\draw [line width=0.6pt, dashed] (13.75,8.75) -- (15,8.75);
\draw [line width=0.6pt, dashed] (26.75,6.75) -- (30,6.75);
\draw [line width=0.6pt, dashed] (26.75,14.5) -- (30,14.5);
\draw [line width=0.6pt, dashed] (26.75,13.75) -- (26.75,12.5);
\draw [line width=0.6pt, dashed] (16.25,13.75) -- (16.25,12.5);
\draw [line width=0.6pt, <->, >=Stealth] (16.25,13) -- (26.75,13)node[pos=0.5,above, fill=white]{100cm};
\draw [line width=0.6pt, dashed] (16.25,16.25) -- (16.25,14.75);
\draw [line width=1pt, ->, >=Stealth] (21.25,17.5) -- (21.25,15)node[pos=0.5,right, fill=white]{P};
\draw [line width=1pt, <->, >=Stealth] (16.25,15.75) -- (21.25,15.75)node[pos=0.5,above, fill=white]{$a$};
\draw [line width=1pt, <->, >=Stealth] (28.75,14.5) -- (28.75,6.75)node[pos=0.5,right, fill=white]{125cm};
\draw [line width=1pt, dashed] (14,14.5) -- (16.25,14.5);
\draw [line width=1pt, <->, >=Stealth] (14.5,14.5) -- (14.5,8.75)node[pos=0.5,left, fill=white]{75cm};
\node [font=\Large] at (13.25,16) {$A$};
\node [font=\Large] at (16,15.5) {$B$};
\node [font=\Large] at (15.5,9.25) {$D$};
\node [font=\Large] at (26.5,15.5) {$C$};
\node [font=\Large] at (25.75,8) {$E$};
\end{circuitikz}
}

        \end{figure}

    %46
    \item An engine working on air standard Otto cycle is supplied with air at $0.1 MPa$ and $35^\degree C$. The compression ratio is $8$. The heat supplied is $500 kJ/kg$. Property data for air: $c_p = 1.005 kJ/kg K,\,c_v = 0.718 kJ/kg K,\,R = 0.287 kJ/kg K$. The maximum temperature (in $K$) of the cycle is \underline{\hspace{1cm}} (correct to one decimal place).

    %47
    \item A plane slab of thickness $L$ and thermal conductivity $k$ is heated with a fluid on one side ($P$),and the other side ($Q$) is maintained at a constant temperature, $T_{Q}$ of $25^\degree C$, as shown in the figure. The fluid is at $45^\degree C$ and the surface heat transfer coefficient, $h$, is $10 W/m^2K$. The steady state temperature, $T_{P}$, (in $^\degree C$) of the side which is exposed to the fluid is \underline{\hspace{1cm}} (correct to two decimal places).
        \begin{figure}[!ht]
            \centering
            \resizebox{0.6\textwidth}{!}{%
\begin{circuitikz}
\tikzstyle{every node}=[font=\Large]
\draw [ line width=1pt ] (12.5,16.25) rectangle (25,8.75);
\draw [line width=1pt, short] (17.5,16.25) -- (17.5,8.75);
\node [font=\Large] at (17.25,16.75) {$A$};
\node [font=\Large] at (25.25,16.75) {$B$};
\node [font=\Large] at (25.25,8.25) {$C$};
\node [font=\Large] at (17.25,8.25) {$D$};
\node [font=\Large] at (12.25,8.25) {$E$};
\node [font=\Large] at (12.25,16.75) {$F$};
\draw [line width=0.6pt, dashed] (12.5,8.75) -- (12.5,6.75);
\draw [line width=0.6pt, dashed] (17.5,8.75) -- (17.5,7);
\draw [line width=0.6pt, dashed] (25,8.75) -- (25,7.25);
\draw [line width=0.6pt, dashed] (25,8.75) -- (26.5,8.75);
\draw [line width=0.6pt, dashed] (25,16.25) -- (26.5,16.25);
\draw [line width=0.6pt, ->, >=Stealth] (20,17.5) .. controls (19,18.75) and (17.5,18.75) .. (16.25,17.5) ;
\node [font=\Large] at (18,19) {$12kNm$};
\draw [line width=0.6pt, <->, >=Stealth] (26,16.25) -- (26,8.75)node[pos=0.5,right, fill=white]{300mm};
\draw [line width=0.6pt, <->, >=Stealth] (17.5,7.75) -- (25,7.75)node[pos=0.5,below, fill=white]{300mm};
\draw [line width=0.6pt, <->, >=Stealth] (17.5,7.75) -- (12.5,7.75)node[pos=0.5,below, fill=white]{200mm};
\end{circuitikz}
}

        \end{figure}

    %48
    \item The true stress $(\sigma)$ - true strain $(\varepsilon)$ diagram of a strain hardening material is shown in figure. First, there is loading up to point $\vec{A}$, i.e., up to stress of $500 MPa$ and strain of $0.5$. Then from point $\vec{A}$, there is unloading up to point $\vec{B}$, i.e., to stress of $100 MPa$. Given that the Young's modulus $E = 200 GPa$, the natural strain at point $\vec{B} (\varepsilon_{B})$ is \underline{\hspace{1cm}} (correct to two decimal places).
        \begin{figure}[!ht]
            \centering
            \resizebox{0.4\textwidth}{!}{%
\begin{circuitikz}
\tikzstyle{every node}=[font=\LARGE]
\draw [line width=1pt, ->, >=Stealth] (5,7.5) -- (5,16.25);
\draw [line width=1pt, ->, >=Stealth] (5,7.5) -- (15,7.5);
\draw [line width=1pt, dashed] (5,15) -- (13,15);
\draw [line width=1pt, dashed] (13,15) -- (13,7.5);
\draw [line width=1pt, dashed] (5,10) -- (12.75,10);
\draw [line width=1pt, dashed] (11.25,10) -- (11.25,7.75);
\draw [line width=1pt, short] (13,15) -- (11.25,10);
\draw [line width=1pt, short] (5,7.5) .. controls (7.75,14.25) and (7.5,13.75) .. (14.5,15.25);
\draw [line width=1pt, short] (8.25,13.75) -- (8.75,13.75);
\draw [line width=1pt, short] (8.75,13.75) -- (8.5,13.25);
\draw [line width=1pt, short] (11.5,11.5) -- (11.5,11);
\draw [line width=1pt, short] (11.5,11) -- (12,11.25);
\node [font=\LARGE] at (11,10.5) {B};
\node [font=\LARGE] at (13,15.5) {A};
\node [font=\LARGE] at (4.25,10) {$100$};
\node [font=\LARGE] at (4.25,15) {$500$};
\node [font=\LARGE] at (5,17.25) {$\sigma$};
\node [font=\LARGE] at (5,16.75) {$(MPa)$};
\node [font=\LARGE] at (13,7) {$0.5$};
\node [font=\LARGE] at (15,7) {$\varepsilon$};
\draw [line width=1pt, short] (5,7) -- (5,6.25);
\draw [line width=1pt, short] (11.25,7) -- (11.25,6.25);
\draw [line width=1pt, <->, >=Stealth] (5,6.75) -- (11.25,6.75)node[pos=0.5, fill=white]{$\varepsilon_{B}$};
\end{circuitikz}
}

        \end{figure}

    %49
    \item An orthogonal cutting operation is being carried out in which uncut thickness is $0.010 mm$, cutting speed is $130 m/min$, rake angle is $15^\degree$ and width of cut is $6 mm$. It is observed that the chip thickness is $0.015 mm$, the cutting force is $60 N$ and the thrust force is $25 N$. The ratio of friction energy to total energy is \underline{\hspace{1cm}} (correct to two decimal places). 

    %50
    \item A bar is compressed to half of its original length. The magnitude of true strain produced in the deformed bar is \underline{\hspace{1cm}} (correct to two decimal places).

    %51
    \item The minimum value of $3x+5y$ such that:
        $$3x+5y\leq15$$
        $$4x+9y\leq8$$
        $$13x+2y\leq2$$
        $$x\geq0,y\geq0$$
        is \underline{\hspace{1cm}}.

    %52
    \item Processing times (including setup times) and due dates for six jobs waiting to be processed at a work centre are given in the table. The average tardiness (in days) using shortest processing time rule is \underline{\hspace{1cm}} (correct to two decimal places).
        \begin{table}[!ht]
            \centering
            \begin{tabular}{|l|c|}
\hline
Month    & Demand \\
\hline
January  & 450    \\
\hline
February & 440    \\
\hline
March    & 460    \\
\hline
April    & 510    \\
\hline
May      & 520    \\
\hline
June     & 495    \\
\hline
July     & 475    \\
\hline
August   & 560    \\
\hline

\end{tabular}

        \end{table}

    %53
    \item The schematic of an external drum rotating clockwise engaging with a short shoe is shown in the figure. The shoe is mounted at point $\vec{Y}$ on a rigid lever $\vec{XYZ}$ hinged at point $\vec{X}$. A force $F = 100 N$ is applied at the free end of the lever as shown. Given that the coefficient offriction between the shoe and the drum is $0.3$, the braking torque (in $Nm$) applied on thedrum is \underline{\hspace{1cm}} (correct to two decimal places).
        \begin{figure}[!ht]
            \centering
            \resizebox{0.6\textwidth}{!}{%
\begin{circuitikz}
\tikzstyle{every node}=[font=\normalsize]
\draw (8,11.25) to[short] (8.25,11.25);
\draw (8,10.75) to[short] (8.25,10.75);
\draw (8.25,11.25) node[ieeestd nor port, anchor=in 1, scale=0.89](port){} (port.out) to[short] (10,11);
\draw (8,8.75) to[short] (8.25,8.75);
\draw (8,8.25) to[short] (8.25,8.25);
\draw (8.25,8.75) node[ieeestd and port, anchor=in 1, scale=0.89](port){} (port.out) to[short] (10,8.5);
\draw (11,11) to[short] (11.25,11);
\draw (11,10.5) to[short] (11.25,10.5);
\draw (11.25,11) node[ieeestd xnor port, anchor=in 1, scale=0.89](port){} (port.out) to[short] (13,10.75);
\draw (11.25,8.5) node[ieeestd not port, anchor=in](port){} (port.out) to[short] (13,8.5);
\draw (port.in) to[short] (11,8.5);
\draw (14.5,10) to[short] (14.75,10);
\draw (14.5,9.5) to[short] (14.75,9.5);
\draw (14.75,10) node[ieeestd or port, anchor=in 1, scale=0.89](port){} (port.out) to[short] (16.5,9.75);
\draw [ line width=0.2pt](10,11) to[short] (11,11);
\draw [ line width=0.2pt](13,10.75) to[short] (13.75,10.75);
\draw [ line width=0.2pt](13.75,10.75) to[short] (13.75,10);
\draw [ line width=0.2pt](13.75,10) to[short] (14.5,10);
\draw [ line width=0.2pt](13,8.5) to[short] (13.75,8.5);
\draw [ line width=0.2pt](13.75,8.5) to[short] (13.75,9.5);
\draw [ line width=0.2pt](13.75,9.5) to[short] (14.5,9.5);
\draw [ line width=0.2pt](10,8.5) to[short] (11,8.5);
\draw [ line width=0.2pt](11,10.5) to[short] (10.5,10.5);
\draw [ line width=0.2pt](10.5,10.5) to[short] (10.5,8.5);
\draw [ line width=0.2pt](8,11.25) to[short] (6.25,11.25);
\draw [ line width=0.2pt](8,8.25) to[short] (6.25,8.25);
\draw [ line width=0.2pt](8,8.75) to[short] (7.5,8.75);
\draw [ line width=0.2pt](8,10.75) to[short] (6.25,10.75);
\draw [ line width=0.2pt](7.5,10.75) to[short] (7.5,8.75);
\node [font=\normalsize] at (17,9.75) {$F$};
\node [font=\normalsize] at (6,11.25) {$A$};
\node [font=\normalsize] at (6,10.75) {$B$};
\node [font=\normalsize] at (6,8.25) {$C$};
\end{circuitikz}
}

        \end{figure}

    %54
    \item Block $P$ of mass $2 kg$ slides down the surface and has a speed $20 m/s$ at the lowest point, $\vec{Q}$, where the local radius of curvature is $2 m$ as shown in the figure. Assuming $g = 10 m/s^2$  , the normal force (in $N$) at $\vec{Q}$ is \underline{\hspace{1cm}} (correct to two decimal places).
        \begin{figure}[!ht]
            \centering
            \resizebox{0.3\textwidth}{!}{%
\begin{circuitikz}
\tikzstyle{every node}=[font=\Huge]
\draw [line width=0.8pt, short] (1.25,16.25) -- (8.75,10);
\draw [line width=0.8pt, short] (1.25,13.75) -- (8.75,7.5);
\draw [line width=0.8pt, short] (8.75,10) .. controls (10.5,8.5) and (12.25,8.25) .. (13.75,10);
\draw [line width=0.8pt, short] (8.75,7.5) .. controls (10.5,6) and (12.25,5.75) .. (13.75,7.5);
\draw [line width=0.8pt, short] (3,15.25) -- (5,13.5);
\draw [line width=0.8pt, short] (5,13.5) -- (6.25,14.75);
\draw [line width=0.8pt, short] (6.25,14.75) -- (4.25,16.5);
\draw [line width=0.8pt, short] (4.25,16.5) -- (3,15.25);
\draw [line width=0.8pt, dashed] (11.25,16.25) -- (11.25,6.25);
\node at (2.75,16.5) {P};
\node at (11.25,5.5) {Q};
\end{circuitikz}
}

        \end{figure}

    %55
    \item An electrochemical machining (ECM) is to be used to cut a through hole into a $12 mm$ thick aluminum plate. The hole has a rectangular cross-section, $10 mm \times 30 mm$. The ECM operation will be accomplished in $2$ minutes, with efficiency of $90\%$. Assuming specific removal rate for aluminum as $3.44 \times 10^{-2} mm^3 /(A s)$, the current (in $A$) required is \underline{\hspace{1cm}} (correct to two decimal places).

    \end{enumerate}

\end{document}
