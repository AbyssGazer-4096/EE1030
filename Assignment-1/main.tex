%iffalse
\let\negmedspace\undefined
\let\negthickspace\undefined
\documentclass[journal,12pt,twocolumn]{IEEEtran}
\usepackage{cite}
\usepackage{amsmath,amssymb,amsfonts,amsthm}
\usepackage{algorithmic}
\usepackage{graphicx}
\usepackage{textcomp}
\usepackage{xcolor}
\usepackage{txfonts}
\usepackage{listings}
\usepackage{enumitem}
\usepackage{mathtools}
\usepackage{gensymb}
\usepackage{comment}
\usepackage[breaklinks=true]{hyperref}
\usepackage{tkz-euclide} 
\usepackage{listings}
\usepackage{gvv}                                        
%\def\inputGnumericTable{}                                 
\usepackage[latin1]{inputenc}                                
\usepackage{color}                                            
\usepackage{array}                                            
\usepackage{longtable}                                       
\usepackage{calc}                                             
\usepackage{multirow}                                         
\usepackage{hhline}                                           
\usepackage{ifthen}                                           
\usepackage{lscape}
\usepackage{tabularx}
\usepackage{array}
\usepackage{float}

\usepackage{multicol}
\usepackage{microtype}

\newtheorem{theorem}{Theorem}[section]
\newtheorem{problem}{Problem}
\newtheorem{proposition}{Proposition}[section]
\newtheorem{lemma}{Lemma}[section]
\newtheorem{corollary}[theorem]{Corollary}
\newtheorem{example}{Example}[section]
\newtheorem{definition}[problem]{Definition}
\newcommand{\BEQA}{\begin{eqnarray}}
\newcommand{\EEQA}{\end{eqnarray}}
\newcommand{\define}{\stackrel{\triangle}{=}}
\theoremstyle{remark}
\newtheorem{rem}{Remark}

\newcommand{\questionA}[6]{
\item #1
    \begin{flushright}
        {\textcolor{magenta}{[#2]}}
    \end{flushright}
    (a) #3\\
    (b) #4\\
    (c) #5\\
    (d) #6\\
}

\newcommand{\questionB}[6]{
\item #1
    \begin{flushright}
        {\textcolor{magenta}{[#2]}}
    \end{flushright}
    \begin{multicols}{2}
        (a) #3
        
        (c) #5
        
        (b) #4
        
        (d) #6
    \end{multicols}
}

\newcommand{\questionC}[6]{
\item #1
    \begin{flushright}
        {\textcolor{magenta}{[#2]}}
    \end{flushright}
    \begin{multicols}{4}
        (a) #3
        
        (b) #4
        
        (c) #5
        
        (d) #6
    \end{multicols}
}

% Marks the beginning of the document
\begin{document}
\bibliographystyle{IEEEtran}
\vspace{3cm}

\title{\textbf{Assignment - 1}}
    \author{\textbf{EE24BTECH11019 - DWARAK A}}
\maketitle
\newpage
\bigskip

\renewcommand{\thefigure}{\theenumi}
\renewcommand{\thetable}{\theenumi}

\section*{\textbf{SECTION-B | JEE Main/AIEEE}}
\bigskip

\begin{enumerate}[label=\textcolor{magenta}{\arabic*.}]

%1
\questionC{A parabola has the origin as its focus and the line $x=2$ as the directrix. Then the vertex of the parabola is at}{2008}
{$(0,2)$}
{$(1,0)$}
{$(0,1)$}
{$(2,0)$}

%2
\questionB{The ellipse $x^2+4y^2=4$ is inscribed in a rectangle aligned with the coordinate axes, which in turn is inscribed in another ellipse that passes through the point $(4,0)$. Then the equation of the ellipse is:}{2009}
{$x^2+12y^2=16$}
{$4x^2+48y^2=48$}
{$4x^2+64y^2=48$}
{$x^2+16y^2=16$}

%3
\questionB{If two tangents drawn from a point P to the parabola $y^2=4x$ are at right angles, then the locus of P is}{2010}
{$2x+1=0$}
{$x=-1$}
{$2x-1=0$}
{$x=1$}

%4
\questionB{Equation of the ellipse whose axes are the axes of coordinates and which passes through the point (-3,1) and has eccentricity $\sqrt{\dfrac{2}{5}}$ is}{2011}
{$5x^2+3y^2-48=0$}
{$3x^2+5y^2-15=0$}
{$5x^2+3y^2-32=0$}
{$3x^2+5y^2-32=0$}

%5
\questionA{\textcolor{magenta}{Statement-1 :} An equation of a common tangent to the parabola $y^2=16\sqrt{3}x$ and the ellipse $2x^2+y^2=4$ is $y=2x+2\sqrt{3}$ 

\textcolor{magenta}{Statement-2 :} If the line $y=mx+\dfrac{4\sqrt{3}}{m},(m\neq0)$ is a common tangent to the parabola $y^2=16\sqrt{3}x$ and the ellipse $2x^2+y^2=4$, then \textit{m} satisfies $m^4+2m^2=24$}{2012}
{Statement-1 is false, Statement-2 is true.}
{Statement-1 is true, Statement-2 is true; Statement-2 is a correct explanation for Statement-1.}
{Statement-1 is true, Statement-2 is true; Statement-2 is \textbf{not} a correct explanation for Statement-1.}
{Statement-1 is true, Statement-2 is false.}

%6
\questionB{An ellipse is drawn by taking a diameter of the circle $(x-1)^2+y^2=1$ as its semi-minor axis and a diameter of the circle $x^2+(y-2)^2=4$ is semi-major axis. If the centre of the ellipse is at the origin and its axes are the coordinate axes, then the equation of the ellipse is: }{2012}
{$4x^2+y^2=4$}
{$x^2+4y^2=8$}
{$4x^2+y^2=8$}
{$x^2+4y^2=16$}

%7
\questionB{The equation of the circle passing through the foci of the ellipse $\dfrac{x^2}{16}+\dfrac{y^2}{9}=1$, and have a center at $(0,3)$ is}{JEE M 2013}
{$x^2+y^2-6y-7=0$}
{$x^2+y^2-6y+7=0$}
{$x^2+y^2-6y-5=0$}
{$x^2+y^2-6y+5=0$}

%8
\questionA{\textcolor{magenta}{Given : }A circle, $2x^2+2y^2=5$ and a parabola, $y^2=4\sqrt{5}x$.

\textcolor{magenta}{Statement-1 :} An equation of a common tangent to these curves is $y=x+\sqrt{5}$.
    
\textcolor{magenta}{Statement-2 :} If the line, $y=mx+\dfrac{\sqrt{5}}{m} (m\neq0)$ is their common tangent, then \textit{m} satisfies $m^4-3m^2+2=0$.}{JEE M 2013}
{Statement-1 is true,Statement-2 is true; Statement-2 is a correct explanation for Statement-1}
{Statement-1 is true,Statement-2 is true; Statement-2 is \textbf{not} a correct explanation for Statement-1}
{Statement-1 is true, Statement-2 is false.}
{Statement-1 is false, Statement-2 is true.}

%9
\questionB{The locus of the foot of perpendicular drawn from the centre of the ellipse $x^2+3y^2=6$ on any tangent to it is}{JEE M 2014}
{$(x^2+y^2)^2=6x^2+2y^2$}
{$(x^2+y^2)^2=6x^2-2y^2$}
{$(x^2-y^2)^2=6x^2+2y^2$}
{$(x^2-y^2)^2=6x^2-2y^2$}

%10
\questionC{The slope of the line touching both the parabolas $y^2=4x$ and $x^2=-32y$ is}{JEE M 2014}
{$\dfrac{1}{8}$}
{$\dfrac{2}{3}$}
{$\dfrac{1}{2}$}
{$\dfrac{3}{2}$}

%11
\questionB{Let O be the vertex and Q be any point on the parabola, $x=8y$. If the point P divides the line segment OQ internally in the ratio 1:3, then locus of P is:}{JEE M 2015}
{$y^2=2x$}
{$x^2=2y$}
{$x^2=y$}
{$y^2=x$}

%12
\questionA{The normal to the curve, $x^2+2xy-3y^2=0$, at $(1,1)$}{JEE M 2015}
{meets the curve again in the third quadrant.}
{meets the curve again in the fourth quadrant.}
{does not meet the curve again.}
{meets the curve again in the second quadrant.}

%13
\questionC{The area (in sq. units) of the quadrilateral formed by the tangents at the end points of the latera recta to the ellipse $\dfrac{x^2}{9}+\dfrac{y^2}{5}=1$, is :}{JEE M 2015}
{$\dfrac{27}{2}$}
{$27$}
{$\dfrac{27}{4}$}
{$18$}

%14
\questionA{Let P be the point on the parabola, $y^2=8x$ which is at a minimum distance from the centre C of the circle, $x^2+(y+6)^2=1$. Then the equation of the circle, passing through C and having its centre at P is:}{JEE M 2016}
{$x^2+y^2-\dfrac{x}{4}+2y-24=0$}
{$x^2+y^2-4x+9y+18=0$}
{$x^2+y^2-4x+8y+12=0$}
{$x^2+y^2-x+4y-12=0$}

%15
\questionC{The eccentricity of the hyperbola whose length of the latus rectum is equal to 8 and the length of its conjugate axis is equal to half of the distance between its foci, is :}{JEE M 2016}
{$\dfrac{2}{\sqrt{3}}$}
{$\sqrt{3}$}
{$\dfrac{4}{3}$}
{$\dfrac{4}{\sqrt{3}}$}

\end{enumerate}

\end{document}