%iffalse
\let\negmedspace\undefined
\let\negthickspace\undefined
\documentclass[journal,12pt,twocolumn]{IEEEtran}
\usepackage{cite}
\usepackage{amsmath,amssymb,amsfonts,amsthm}
\usepackage{algorithmic}
\usepackage{graphicx}
\usepackage{textcomp}
\usepackage{xcolor}
\usepackage{txfonts}
\usepackage{listings}
\usepackage{enumitem}
\usepackage{mathtools}
\usepackage{gensymb}
\usepackage{comment}
\usepackage[breaklinks=true]{hyperref}
\usepackage{tkz-euclide} 
\usepackage{listings}
\usepackage{gvv}                                        
%\def\inputGnumericTable{}                                 
\usepackage[latin1]{inputenc}                                
\usepackage{color}                                            
\usepackage{array}                                            
\usepackage{longtable}                                       
\usepackage{calc}                                             
\usepackage{multirow}                                         
\usepackage{hhline}                                           
\usepackage{ifthen}                                           
\usepackage{lscape}
\usepackage{tabularx}
\usepackage{array}
\usepackage{float}

\usepackage{enumitem}
\usepackage{xcolor}
%\usepackage{multicol}


\newtheorem{theorem}{Theorem}[section]
\newtheorem{problem}{Problem}
\newtheorem{proposition}{Proposition}[section]
\newtheorem{lemma}{Lemma}[section]
\newtheorem{corollary}[theorem]{Corollary}
\newtheorem{example}{Example}[section]
\newtheorem{definition}[problem]{Definition}
\newcommand{\BEQA}{\begin{eqnarray}}
\newcommand{\EEQA}{\end{eqnarray}}
\newcommand{\define}{\stackrel{\triangle}{=}}
\theoremstyle{remark}
\newtheorem{rem}{Remark}

% Marks the beginning of the document
\begin{document}
\bibliographystyle{IEEEtran}
\vspace{3cm}

\title{SECTION B (Q16-Q30)}
\author{EE24BTECH11019 - Dwarak A}
\maketitle
\newpage
\bigskip

\renewcommand{\thefigure}{\theenumi}
\renewcommand{\thetable}{\theenumi}

\begin{enumerate}[label=\textcolor{magenta}{\arabic*.}]
%1
    \item A parabola has the origin as its focus and the line x = 2 as the directrix. Then the vertex of the parabola is at
    \hfill{\textcolor{magenta}{[2008]}}

    \begin{tabbing}
        (a) $(0,2)$ \hspace{.8em} \= (b) $(1,0)$ \hspace{.8em} \= (c) $(0,1)$ \hspace{.8em} \= (d) $(2,0)$ \\
    \end{tabbing}
%2
    \item The ellipse $x^2+4y^2=4$ is inscribed in a rectangle aligned with the coordinate axes, which in turn is inscribed in another ellipse that passes through the point (4,0). Then the equation of the ellipse is:
    \hfill{\textcolor{magenta}{[2009]}}

    \begin{tabbing}
        (a) $x^2+12y^2=16$ \hspace{1.2em} \= (b) $4x^2+48y^2=48$ \\
        (c) $4x^2+64y^2=48$ \hspace{0.7em} \= (d) $x^2+16y^2=16$ \\
    \end{tabbing}
%3
    \item If two tangents drawn from a point P to the parabola $y^2=4x$ are at right angles, then the locus of P is
    \hfill{\textcolor{magenta}{[2010]}}
    
    \begin{tabbing}
        (a) $2x+1=0$ \hspace{3.2em} \= (b) $x=-1$ \\
        (c) $2x-1=0$ \hspace{3.2em} \= (d) $x=1$ \\
    \end{tabbing}
%4
    \item Equation of the ellipse whose axes are the axes of coordinates and which passes through
    the point (-3,1) and has eccentricity $\sqrt{\frac{2}{5}}$
    \hfill{\textcolor{magenta}{[2011]}}
    
    \begin{tabbing}
        (a) $5x^2+3y^2-48=0$ \hspace{0em} \= (b) $3x^2+5y^2-15=0$ \\
        (c) $5x^2+3y^2-32=0$ \hspace{0em} \= (d) $3x^2+5y^2-32=0$ \\
    \end{tabbing}
%5
    \item
    \textcolor{magenta}{Statement-1 :} An equation of a common tangent to the parabola $y^2=16\sqrt{3}x$ and the ellipse $2x^2+y^2=4$ is $y=3x+2\sqrt{3}$
    
    \textcolor{magenta}{Statement-2 :} If the line $y=mx+\frac{4\sqrt{3}}{m},(m\neq0)$ is a common tangent to the parabola $y^2=16\sqrt{3}x$ and the ellipse $2x^2+y^2=4$,then \textit{m} satisfies $m^4+2m^2=24$ 
    \hfill{\textcolor{magenta}{[2012]}}
    
    (a) Statement-1 is false,Statement-2 is true.\\
    (b) Statement-1 is true,Statement-2 is true; Statement-2
    is a correct explanation for Statement-1\\
    (c) Statement-1 is true,Statement-2 is true; Statement-2
    is \textbf{not} a correct explanation for Statement-1\\
    (d) Statement-1 is true,Statement-2 is false.\\
%6  
    \item An ellipse is drawn by taking a diameter of the circle $(x-1)^2+y^2=1$ as its semi-minor axis and a diameter of the circle $x2+(y-2)^22=4$ is semi-major axis. If the centre of the ellipse is at the origin and its axes are the coordinate axes, then the equation of the ellipse is: 
    \hfill{\textcolor{magenta}{[2012]}}
    
    \begin{tabbing}
        (a) $4x^2+y^2=4$ \hspace{3em} \= (b) $x^2+4y^2=8$ \\
        (c) $4x^2+y^2=8$ \hspace{3em} \= (d) $x^2+4y^2=16$ \\
    \end{tabbing}
%7
    \item The equation of the circle passing through the foci of the ellipse $\frac{x^2}{16}+\frac{y^2}{9}=1$, and have a center at $(0,3)$ is
    
    \hfill{\textcolor{magenta}{[JEE M 2013]}}
    
    \begin{tabbing}
        (a) $x^2+y^2-6y-7=0$ \hspace{0em} \= (b) $x^2+y^2-6y+7=0$ \\
        (c) $x^2+y^2-6y-5=0$ \hspace{0em} \= (d) $x^2+y^2-6y+5=0$ \\
    \end{tabbing}
%8
    \item
    \textcolor{magenta}{Given : }A circle,$2x^2+2y^2=5$ and a parabola, $y^2=4\sqrt{5}x$.
    
    \textcolor{magenta}{Statement-1 :} An equation of a common tangent to these curves is $y=x+\sqrt{5}$.
    
    \textcolor{magenta}{Statement-2 :} If the line, $y=mx+\frac{\sqrt{5}}{m} (m\neq0)$ is their common tangent, then \textit{m} satisfies $m^4-3m^2+2=0$.

    \hfill{\textcolor{magenta}{[JEE M 2013]}}
    
    (a) Statement-1 is true,Statement-2 is true; Statement-2
    is a correct explanation for Statement-1\\
    (b) Statement-1 is true,Statement-2 is true; Statement-2
    is \textbf{not} a correct explanation for Statement-1\\
    (c) Statement-1 is true,Statement-2 is false.\\
    (d) Statement-1 is false,Statement-2 is true.\\
%9
    \item The locus of the foot of perpendicular drawn from the centre of the ellipse $x^2+3y^2=6$ on any tangent to it is
    
    \hfill{\textcolor{magenta}{[JEE M 2014]}}
    
    \begin{tabbing}
        (a) $(x^2+y^2)^2=6x^2+2y^2$ \hspace{0em} \= (b) $(x^2+y^2)^2=6x^2-2y^2$ \\
        (c) $(x^2-y^2)^2=6x^2+2y^2$ \hspace{0em} \= (d) $(x^2-y^2)^2=6x^2-2y^2$ \\
    \end{tabbing}

    \newpage
%10
    \item The slope of the line touching both the parabolas $y^2=4x$ and $x^2=-32y$

    \hfill{\textcolor{magenta}{[JEE M 2014]}}
    \begin{tabbing}
        (a) $\frac{1}{8}$ \hspace{2em} \= (b) $\frac{2}{3}$ \hspace{2em} \= (c) $\frac{1}{2}$ \hspace{2em} \= (d) $\frac{3}{2}$ \\
    \end{tabbing}
%11
    \item Let O be the vertex and Q be any point on the parabola, $x=8y$. If the point P divides the line segment OQ internally in the ratio 1:3, then locus of P is:
    \hfill{\textcolor{magenta}{[JEE M 2015]}}
    \begin{tabbing}
        (a) $y^2=2x$ \hspace{0em} \= (b) $x^2=2y$ \hspace{0em} \= (c) $x^2=y$ \hspace{0em} \= (d) $y^2=x$ \\
    \end{tabbing}
%12
    \item The normal to the curve, $x^2+2xy-3y^2=0$, at $(1,1)$
    \hfill{\textcolor{magenta}{[JEE M 2015]}}

    (a) meets the curve again in the third quadrant.\\
    (b) meets the curve again in the fourth quadrant.\\
    (c) does not meet the curve again.\\
    (d) meets the curve again in the second quadrant.\\
%13
    \item The area (in sq. units) of the quadrilateral formed by the tangents at the end points of the latera recta to the ellipse $\frac{x^2}{9}+\frac{y^2}{5}=1$, is :
    
    \hfill{\textcolor{magenta}{[JEE M 2015]}}
    \begin{tabbing}
        (a) $\frac{27}{2}$ \hspace{2em} \= (b) $27$ \hspace{2em} \= (c) $\frac{27}{4}$ \hspace{2em} \= (d) 18 \\
    \end{tabbing}
%14
    \item Let P be the point on the parabola, $y^2=8x$ which is at a minimum distance from the centre C of the circle, $x^2+(y+6)^2=1$. Then the equation of the circle, passing through C and having its centre at P is:
    \hfill{\textcolor{magenta}{[JEE M 2016]}}
    (a) $x^2+y^2-\frac{x}{4}+2y-24=0$\\
    (b) $x^2+y^2-4x+9y+18=0$\\
    (c) $x^2+y^2-4x+8y+12=0$\\
    (d) $x^2+y^2-4x+4y-12=0$\\
%15
    \item The eccentricity of the hyperbola whose length of the latus rectum is equal to 8 and the length of its conjugate axis is equal to half of the distance between its foci, is :
    \hfill{\textcolor{magenta}{[JEE M 2016]}}
    \begin{tabbing}
        (a) $\frac{2}{\sqrt{3}}$ \hspace{2em} \= (b) $\sqrt{3}$ \hspace{2em} \= (c) $\frac{4}{3}$ \hspace{2em} \= (d)$\frac{4}{\sqrt{3}}$\\
    \end{tabbing}
    
\end{enumerate}

\end{document}