%iffalse
\let\negmedspace\undefined
\let\negthickspace\undefined
\documentclass[journal,12pt,twocolumn]{IEEEtran}
\usepackage{cite}
\usepackage{amsmath,amssymb,amsfonts,amsthm}
\usepackage{algorithmic}
\usepackage{graphicx}
\usepackage{textcomp}
\usepackage{xcolor}
\usepackage{txfonts}
\usepackage{listings}
\usepackage{enumitem}
\usepackage{mathtools}
\usepackage{gensymb}
\usepackage{comment}
\usepackage[breaklinks=true]{hyperref}
\usepackage{tkz-euclide} 
\usepackage{listings}
\usepackage{gvv}                                        
%\def\inputGnumericTable{}                                 
\usepackage[latin1]{inputenc}                                
\usepackage{color}                                            
\usepackage{array}                                            
\usepackage{longtable}                                       
\usepackage{calc}                                             
\usepackage{multirow}                                         
\usepackage{hhline}                                           
\usepackage{ifthen}                                           
\usepackage{lscape}
\usepackage{tabularx}
\usepackage{array}
\usepackage{float}

\newtheorem{theorem}{Theorem}[section]
\newtheorem{problem}{Problem}
\newtheorem{proposition}{Proposition}[section]
\newtheorem{lemma}{Lemma}[section]
\newtheorem{corollary}[theorem]{Corollary}
\newtheorem{example}{Example}[section]
\newtheorem{definition}[problem]{Definition}
\newcommand{\BEQA}{\begin{eqnarray}}
\newcommand{\EEQA}{\end{eqnarray}}
\newcommand{\define}{\stackrel{\triangle}{=}}
\theoremstyle{remark}
\newtheorem{rem}{Remark}

% Marks the beginning of the document
\begin{document}
\bibliographystyle{IEEEtran}
\vspace{3cm}

\title{\textbf{ASSIGNMENT - 1}}
\author{\textbf{EE24BTECH11019 - Dwarak A}}
\maketitle
\newpage
\bigskip

\renewcommand{\thefigure}{\theenumi}
\renewcommand{\thetable}{\theenumi}

\section*{\textbf{SECTION B | JEE MAIN/AIEEE}}
\bigskip

\begin{enumerate}
%16
    \item If $a_1,a_2,\dots,a_n$ are in H.P., then the expression $a_1a_2+a_2a_3+\dots+a_{n-1}a_n$ is equal to
    
    \hfill(2006)

    \begin{enumerate}
    \item$n(a_1-a_n)$
    \item$(n-1)(a_1-a_n)$
    \item$na_1a_n$
    \item$(n-1)a_1a_n$ 
    \end{enumerate}
%17
    \item The sum of series $\frac{1}{2!}-\frac{1}{3!}+\frac{1}{4!}-\dots$ upto infinity is 
    \hfill(2007)

    \begin{enumerate}
    \item$e^{-\frac{1}{2}}$
    \item$e^{+\frac{1}{2}}$
    \item$e^{-2}$
    \item$e^{-1}$
    \end{enumerate}
%18
    \item  In a geometric progression consisting of positive terms, each term equals the sum of the next two terms. Then the common ratio of its progression is equals   
    \hfill(2007)
    
    \begin{enumerate}
    \item$\sqrt{5}$
    \item$\frac{1}{2}\brak{\sqrt{5}-1}$
    \item$\frac{1}{2}\brak{1-\sqrt{5}}$
    \item$\frac{1}{2}\sqrt{5}$ 
    \end{enumerate}
%19
    \item  The first two terms of a geometric progression add up to $12$. the sum of the third and the fourth terms is $48$. If the terms of the geometric progression are alternately positive and negative, then the first term is   
    \hfill(2008)
    
    \begin{enumerate}
    \item$-4$
    \item$-12$
    \item$12$ 
    \item$4$ 
    \end{enumerate}
%20
    \item The sum to infinite term of the series $1+\frac{2}{3}+\frac{6}{3^2}+\frac{10}{3^3}+\frac{14}{3^4}+\dots$ is
    \hfill(2009)
    \begin{enumerate}
    \item $3$
    \item $4$
    \item $6$
    \item $2$
    \end{enumerate}
%21  
    \item  A person is to count $4500$ currency notes. Let $a_n$ denote the number of notes he counts in the $n^{th}$ minute. If $a_1=a_2=\dots=a_{10}=150$ and $a_{10},a_{11},\dots$ are in an AP with common difference $-2$, then the time taken by him to count all notes is   
     
    \hfill(2010)
    
    \begin{enumerate}
    \item$34$ minutes
    \item$125$ minutes
    \item$135$ minutes
    \item$24$ minutes 
    \end{enumerate}
%22
    \item A man saves Rs.$200$ in each of the first three months of his service. In each of the subsequent months his saving increases by Rs.$40$ more than the saving of immediately previous month. His total saving from the start of service will be Rs.$11040$ after    
    
    \hfill(2011)
    
    \begin{enumerate}
    \item$19$ months
    \item$20$ months
    \item$21$ months
    \item$18$ months

    \end{enumerate}
%23
    \item 
    Statement-1 : The sum of the series $1+(1+2+4)+(4+6+9)+(9+12+16)+\dots+(361+380+400)$ is $8000$.    
    
    Statement-2 : $\sum\limits_{k=1}^n\brak{k^3-\brak{k-1}^3}=n^3$, for any natural number $n$.
    \hfill(2012)
    \begin{enumerate}
    \item Statement-1 is false, Statement-2 is true.
    \item Statement-1 is true; Statement-2 is true; Statement-2 is a correct explanation for Statement-1
    \item Statement-1 is true; Statement-2 is true; Statement-2 is not a correct explanation for Statement-1
    \item Statement-1 is true; Statement-2 is false.
    \end{enumerate}
%24
    \item The sum of first $20$ terms of the sequence $0.7,0.77,0.777,\dots$ is 
    \hfill(JEE M 2013)
    
    \begin{enumerate}
    \item$\frac{7}{81}\brak{179-10^{-20}}$
    \item$\frac{7}{9}\brak{99-10^{-20}}$
    \item$\frac{7}{81}\brak{179+10^{-20}}$
    \item$\frac{7}{9}\brak{99+10^{-20}}$
    \end{enumerate}
%25
    \item If $(10)^9+2(11)^1(10)^8+3(11)^2(10)^7+\dots+10(11)^9=k(10)^9$, then $k$ is equal to:
    
    \hfill(JEE M 2014)
    \begin{enumerate}
    \item$100$
    \item$110$
    \item$\frac{121}{10}$
    \item$\frac{441}{100}$ 
    \end{enumerate}
%26
    \item Three positive numbers form an increasing G.P. If the middle term in this G.P. is doubled, the new numbers are in A.P. then the common ratio of the G.P. is: 
    
    \hfill(JEE M 2014)
    \begin{enumerate}
    \item$2-\sqrt{3}$
    \item$2+\sqrt{3}$
    \item$\sqrt{2}+\sqrt{3}$
    \item$3+\sqrt{2}$ 
    \end{enumerate}
%27
    \item The sum of first $9$ terms of the series.
	    $\frac{1^3}{1}+\frac{1^3+2^3}{1+3}+\frac{1^3+2^3+3^3}{1+3+5}+\dots$
    \hfill(JEE M 2015)
    \begin{enumerate}
    \item $142$
    \item $192$
    \item $71$
    \item $96$
    \end{enumerate}
%28
    \item If $m$ is the A.M. of two distinct real numbers $l$ and $n$ $(l,n>1)$ and $G_1$, $G_2$ and $G_3$ are three geometric means between $l$ and $n$, then $G_1^4+2G_2^4+G_3^4$ equals :
    
    \hfill(JEE M 2015)
    \begin{enumerate}
    \item$4lmn^2$
    \item$4l^2m^2n^2$
    \item$4l^2mn$
    \item$4lm^2n$ 
    \end{enumerate}
%29
    \item If the $2^{nd}$, $5^{th}$ and $9^{th}$ terms of a non-constant A.P. are in G.P., then the common ratio of this G.P. is:
    \hfill(JEE M 2016)
    \begin{enumerate}
    \item $1$
    \item $\frac{7}{4}$
    \item $\frac{8}{5}$
    \item $\frac{4}{3}$
    \end{enumerate}
%30
    \item If the sum of the first ten terms of the series $\brak{1\frac{3}{5}}^2+\brak{2\frac{2}{5}}^2+\brak{3\frac{1}{5}}^2+4^2+\brak{4\frac{4}{5}}^2+\dots$, is $\frac{16}{5}m$, then $m$ is equal to :
    
    \hfill(JEE M 2016)
    \begin{enumerate}
    \item$100$
    \item$99$
    \item$102$
    \item$101$
    \end{enumerate}
\end{enumerate}

\end{document}
