%iffalse
\let\negmedspace\undefined
\let\negthickspace\undefined
\documentclass[journal,12pt,twocolumn]{IEEEtran}
\usepackage{cite}
\usepackage{amsmath,amssymb,amsfonts,amsthm}
\usepackage{algorithmic}
\usepackage{graphicx}
\usepackage{textcomp}
\usepackage{xcolor}
\usepackage{txfonts}
\usepackage{listings}
\usepackage{enumitem}
\usepackage{mathtools}
\usepackage{gensymb}
\usepackage{comment}
\usepackage[breaklinks=true]{hyperref}
\usepackage{tkz-euclide} 
\usepackage{listings}
\usepackage{gvv}                                        
%\def\inputGnumericTable{}                                 
\usepackage[latin1]{inputenc}                                
\usepackage{color}                                            
\usepackage{array}                                            
\usepackage{longtable}                                       
\usepackage{calc}                                             
\usepackage{multirow}                                         
\usepackage{hhline}                                           
\usepackage{ifthen}                                           
\usepackage{lscape}
\usepackage{tabularx}
\usepackage{array}
\usepackage{float}

\newtheorem{theorem}{Theorem}[section]
\newtheorem{problem}{Problem}
\newtheorem{proposition}{Proposition}[section]
\newtheorem{lemma}{Lemma}[section]
\newtheorem{corollary}[theorem]{Corollary}
\newtheorem{example}{Example}[section]
\newtheorem{definition}[problem]{Definition}
\newcommand{\BEQA}{\begin{eqnarray}}
\newcommand{\EEQA}{\end{eqnarray}}
\newcommand{\define}{\stackrel{\triangle}{=}}
\theoremstyle{remark}
\newtheorem{rem}{Remark}

% Marks the beginning of the document
\begin{document}
\bibliographystyle{IEEEtran}
\vspace{3cm}

\title{\textbf{JEE MAIN 2023\\JANUARY 29, SHIFT-2}}
\author{EE24BTECH11019 - Dwarak A}
\maketitle
\newpage
\bigskip

\renewcommand{\thefigure}{\theenumi}
\renewcommand{\thetable}{\theenumi}

\section*{\textbf{SECTION-A}}
\bigskip

\begin{enumerate}
    %1
    \item The statement $B\implies((\sim A)\vee B)$ is equivalent to : 
        \begin{enumerate}
            \item $B\implies(A \implies B)$
            \item $A \implies(A \iff B)$
            \item $A \implies((\sim A)\implies B)$
            \item $B \implies((\sim A)\implies B)$
        \end{enumerate}
    
    %2
    \item The shortest distance between the lines $$\frac{x-1}{2}=\frac{y+8}{-7}=\frac{z-4}{5} \text{ and } \frac{x-1}{2}=\frac{y-2}{1}=\frac{z-6}{-3}$$ is
        \begin{enumerate}
            \item $2\sqrt{3}$
            \item $4\sqrt{3}$
            \item $3\sqrt{3}$
            \item $5\sqrt{3}$
        \end{enumerate}
    
    %3
    \item If $\vec{a}=\hat{i}+2\hat{k},\vec{b}=\hat{i}+\hat{j}+\hat{k},\vec{c}=7\hat{i}-3\hat{j}+4\hat{k},\vec{r}\times\vec{b}+\vec{b}\times\vec{c}=\vec{0}$ and $\vec{r}\cdot\vec{a}=0$ then $\vec{r}\cdot\vec{c}$ is equal to :
        \begin{enumerate}
            \item $34$
            \item $12$
            \item $36$
            \item $30$
        \end{enumerate}

    %4
    \item Let $S=\cbrak{w_1,w_2,\dots}$ be the sample space associated to a random experiment. Let $P(w_n)=\frac{P(w_{n-1})}{2},n\geq2$. Let $A=\cbrak{2k+3l;k,l\in \mathbb{N}}$ and $B=\cbrak{w_n;n\in A}$. Then $P(B)$ is equal to
        \begin{enumerate}
            \item $\frac{3}{32}$
            \item $\frac{3}{64}$
            \item $\frac{1}{16}$
            \item $\frac{1}{32}$
        \end{enumerate}

    %5
    \item The value of the integral $\int\limits_1^2\brak{\frac{t^4+1}{t^6+1}}dt$ is :
        \begin{enumerate}
            \item $\tan^{-1}\frac{1}{2}+\frac{1}{3}\tan^{-1}8-\frac{\pi}{3}$
            \item $\tan^{-1}2-\frac{1}{3}\tan^{-1}8+\frac{\pi}{3}$
            \item $\tan^{-1}2+\frac{1}{3}\tan^{-1}8-\frac{\pi}{3}$
            \item $\tan^{-1}\frac{1}{2}-\frac{1}{3}\tan^{-1}8+\frac{\pi}{3}$
        \end{enumerate}

    %6
    \item Let $K$ be the sum of the coefficients of the odd powers of $x$ in the expansion of $(1+x)^{99}$. Let $a$ be the middle term in the expansion of $\brak{2+\frac{1}{\sqrt{2}}}^{200}$. If $\frac{\comb{200}{99}K}{a}=\frac{2^lm}{n}$, where $m$ and $n$ are odd numbers, then the ordered pair $(l,n)$ is equal to :
        \begin{enumerate}
            \item $(50,51)$
            \item $(51,99)$
            \item $(50,101)$
            \item $(51,101)$
        \end{enumerate}

    %7
    \item Let $f$ and $g$ be twice differentiable functions on $\mathbb{R}$ such that
        $$f^{\prime\prime}(x)=g^{\prime\prime}(x)+6x$$
        $$f^\prime(1)=4g^\prime(1)-3=9$$
        $$f(2)=3g(2)=12$$
        Then which of the following is NOT true ?
        \begin{enumerate}
            \item $g(-2)-f(-2)=20$
            \item If $-1<x<2$, then $\abs{f(x)-g(x)}<8$
            \item $\abs{f^\prime(x)-g^\prime(x)}<6\implies-1<x<1$
            \item There exists $x_0\in\brak{1,\frac{3}{2}}$ such that $f(x_0)=g(x_0)$
        \end{enumerate}


    %8
    \item The set of all values of $t\in\mathbb{R}$, for which the matrix $$\myvec{e^t & e^{-t}(\sin{t}-2\cos{t}) & e^{-t}(-2\sin{t}-\cos{t}) \\ e^t & e^{-t}(2\sin{t}+\cos{t}) & e^{-t}(\sin{t}-2\cos{t}) \\ e^t & e^{-t}\cos{t} & e^{-t}\sin{t}}$$ is invertible, is :
        \begin{enumerate}
            \item $\cbrak{(2k+1)\frac{\pi}{2},k\in\mathbb{Z}}$
            \item $\cbrak{k\pi+\frac{\pi}{4},k\in\mathbb{Z}}$
            \item $\cbrak{k\pi,k\in\mathbb{Z}}$
            \item $\mathbb{R}$
        \end{enumerate}


    %9
    \item The area of the region $$A=\cbrak{(x,y):\abs{\cos x-\sin x}\leq y\leq\sin x, 0\leq x\leq\frac{\pi}{2}}$$
        \begin{enumerate}
            \item $1 - \frac{3}{\sqrt{2}} + \frac{4}{\sqrt{5}}$
            \item $\sqrt{5} + 2\sqrt{2} - 4.5$
            \item $\frac{3}{\sqrt{5}} - \frac{3}{\sqrt{2}} + 1$
            \item $\sqrt{5} - 2\sqrt{2} + 1$
        \end{enumerate}

    %10
    \item The set of all values of $\lambda$ for which the equation $\cos^22x-2\sin^4x-2\cos^2x=\lambda$ 
        \begin{enumerate}
            \item \sbrak{-2,-1}
            \item \sbrak{-2,-\frac{3}{2}}
            \item \sbrak{-1,-\frac{1}{2}}
            \item \sbrak{-\frac{3}{2},-1}
        \end{enumerate}

    %11
    \item The letters of the word OUGHT are written in all possible ways and these words are arranged as in a dictionary, in a series. Then the serial number of the word TOUGH is :
        \begin{enumerate}
            \item $89$
            \item $84$
            \item $86$
            \item $79$
        \end{enumerate}

    %12
    \item The plane $2x-y+z=4$ intersects the line segment joining the points $\vec{A}(a, -2, 4)$ and $\vec{B}(2, b, -3)$ at the point $\vec{C}$ in the ratio $2:1$ and the distance of the point $\vec{C}$ from the origin is $\sqrt{5}$. If $ab<0$ and $\vec{P}$ is the point $(a-b,b,2b-a)$ then $\vec{CP}^2$ is equal to :
        \begin{enumerate}
            \item $\frac{17}{3}$
            \item $\frac{16}{3}$
            \item $\frac{73}{3}$
            \item $\frac{97}{3}$
        \end{enumerate}

    %13
    \item Let $\vec{a} = 4\hat{i} + 3\hat{j}$ and $\vec{b} = 3\hat{i} - 4\hat{j} + 5\hat{k}$ and $\vec{c}$ is a vector such that $\vec{c} \cdot (\vec{a} \times \vec{b}) + 25 = 0, \quad \vec{c} \cdot (\hat{i} + \hat{j} + \hat{k}) = 4$ and projection of $\vec{c}$ on $\vec{a}$ is $1$, then the projection of $\vec{c}$ on $\vec{b}$ equals:
        \begin{enumerate}
            \item $\frac{5}{\sqrt{2}}$
            \item $\frac{1}{5}$
            \item $\frac{1}{\sqrt{2}}$
            \item $\frac{3}{\sqrt{2}}$
        \end{enumerate}

    %14
    \item If the lines $\frac{x-1}{1} = \frac{y-2}{2} = \frac{z+3}{1}$ and $\frac{x-a}{2} = \frac{y+2}{3} = \frac{z-3}{1}$ intersect at the point $\vec{P}$, then the distance of the point $\vec{P}$ from the plane $z=a$ is :
        \begin{enumerate}
            \item $16$
            \item $28$
            \item $10$
            \item $22$
        \end{enumerate}

    %15
    \item The value of the integral $\int\limits_\frac{1}{2}^2\frac{\tan^{-1}x}{x}dx$ is equal to :
        \begin{enumerate}
            \item $\pi\log_e2$
            \item $\frac{1}{2}\log_e2$
            \item $\frac{\pi}{4}\log_e2$
            \item $\frac{\pi}{2}\log_e2$
        \end{enumerate}

\end{enumerate}

\end{document}
