%iffalse
\let\negmedspace\undefined
\let\negthickspace\undefined
\documentclass[journal,12pt,twocolumn]{IEEEtran}
\usepackage{cite}
\usepackage{amsmath,amssymb,amsfonts,amsthm}
\usepackage{algorithmic}
\usepackage{graphicx}
\usepackage{textcomp}
\usepackage{xcolor}
\usepackage{txfonts}
\usepackage{listings}
\usepackage{enumitem}
\usepackage{mathtools}
\usepackage{gensymb}
\usepackage{comment}
\usepackage[breaklinks=true]{hyperref}
\usepackage{tkz-euclide} 
\usepackage{listings}
\usepackage{gvv}                                        
%\def\inputGnumericTable{}                                 
\usepackage[latin1]{inputenc}                                
\usepackage{color}                                            
\usepackage{array}                                            
\usepackage{longtable}                                       
\usepackage{calc}                                             
\usepackage{multirow}                                         
\usepackage{hhline}                                           
\usepackage{ifthen}                                           
\usepackage{lscape}
\usepackage{tabularx}
\usepackage{array}
\usepackage{float}

\newtheorem{theorem}{Theorem}[section]
\newtheorem{problem}{Problem}
\newtheorem{proposition}{Proposition}[section]
\newtheorem{lemma}{Lemma}[section]
\newtheorem{corollary}[theorem]{Corollary}
\newtheorem{example}{Example}[section]
\newtheorem{definition}[problem]{Definition}
\newcommand{\BEQA}{\begin{eqnarray}}
\newcommand{\EEQA}{\end{eqnarray}}
\newcommand{\define}{\stackrel{\triangle}{=}}
\theoremstyle{remark}
\newtheorem{rem}{Remark}

% Marks the beginning of the document
\begin{document}
\bibliographystyle{IEEEtran}
\vspace{3cm}

\title{\textbf{JEE MAIN 2020\\JANUARY 7, SHIFT-2}}
\author{EE24BTECH11019 - Dwarak A}
\maketitle
\newpage
\bigskip

\renewcommand{\thefigure}{\theenumi}
\renewcommand{\thetable}{\theenumi}

\section*{\textbf{SECTION-A}}
\bigskip

\begin{enumerate}

        %1
    \item Let $A=[a_{ij}]$ and $B=[b_{ij}]$ be two $3\times3$ real matrices such that $b_{ij}-(3)^{(i+j-2)}a_{ji}$, where $i,j=1,2,3$. If the determinant of $B$ is $81$, then the determinant of $A$ is :
        \begin{enumerate}
            \item $\frac{1}{3}$
            \item $\frac{1}{9}$
            \item $\frac{1}{81}$
            \item $3$
        \end{enumerate}

        %2
    \item The locus of mid points of the perpendiculars drawn from points on the line, $x=2y$ to the line $x=y$ is :
        \begin{enumerate}
            \item $3x-2y=0$
            \item $2x-3y=0$
            \item $7x-5y=0$
            \item $5x-7y=0$
        \end{enumerate}

        %3
    \item Let the tangents drawn from the origin to the circle, $x^2+y^2-8x-4y+16=0$ touch it at the points $A$ and $B$. Then $\brak{\vec{AB}}^2$ is equal to:
        \begin{enumerate}
            \item $\frac{32}{5}$
            \item $\frac{52}{5}$
            \item $\frac{56}{5}$
            \item $\frac{64}{5}$
        \end{enumerate}

        %4
    \item Let $A$, $B$, $C$ and $D$ be four non-empty sets. The Contrapositive statement of "If $A \subseteq B$ and $B \subseteq D$, then $A \subseteq C$" is :
        \begin{enumerate}
            \item If $A \nsubseteq C$, then $A \nsubseteq B$ or $B \subseteq D$
            \item If $A \nsubseteq C$, then $A \nsubseteq B$ and $B \nsubseteq D$
            \item If $A \nsubseteq C$, then $A \subseteq B$ and $B \subseteq D$
            \item If $A \subseteq C$, then $B \subset A$ or $D \subset B$
        \end{enumerate}

        %5
    \item Let $y=y(x)$ be the solution curve of the differential equation $\brak{y^2-x}\frac{dy}{dx}=1$ satisfying $y(0)=1$. This curve intersects the x-axis at a point whose abscissa is :
        \begin{enumerate}
            \item $2$
            \item $2+e$
            \item $2-e$
            \item $-e$
        \end{enumerate}

        %6
    \item If $\theta_1$ and $\theta_2$ be respectively the smallest and largest values of $\theta$ in $\brak{0,2\pi}-\cbrak{\pi}$ which satisfy the equation $2\cot^2\theta-\frac{5}{\sin\theta}+4=0$ then $\int_{\theta_1}^{\theta_2}\cos^23\theta\,d\theta$ is equal to :
        \begin{enumerate}
            \item $\frac{\pi}{3}+\frac{1}{6}$
            \item $\frac{\pi}{9}$
            \item $\frac{\pi}{3}$
            \item $\frac{2\pi}{3}$
        \end{enumerate}

        %7
    \item If the sum of the first $40$ terms of the series, $3+4+8+9+13+14+18+19+\dots$ is $(102)m$, then $m$ is equal to :
        \begin{enumerate}
            \item $25$
            \item $20$
            \item $10$
            \item $5$
        \end{enumerate}

        %8
    \item The number of ordered pairs \brak{r,k} for which $6\cdot\comb{35}{r}=(k^2-3)\cdot\comb{36}{r+1}$, where k is an integer, is :
        \begin{enumerate}
            \item $6$
            \item $4$
            \item $3$
            \item $2$
        \end{enumerate}

        %9
    \item The value of $\alpha$ for which $4\alpha\int\limits_{-1}^{2}e^{-\alpha\abs{x}}\,dx=5$ is: 
        \begin{enumerate}
            \item $\log_e\brak{\frac{4}{3}}$
            \item $\log_e2$
            \item $\log_e\sqrt{2}$
            \item $\log_e\brak{\frac{3}{2}}$
        \end{enumerate}

        %10
    \item Let $f(x)$ be a polynomial of degree $5$ such that $x=\pm1$ are its critical points. If $\lim\limits_{x\to0}\brak{2+\frac{f(x)}{x^3}}=4$ then which of the following is not true ?
        \begin{enumerate}
            \item $f$ is an odd function
            \item $x=1$ is a point of maxima and $x=-1$ is a point of minima of $f$.
            \item $f(1)-4f(-1)=4$
            \item $x=1$ is a point of minima and $x=-1$ is a point of maxima of $f$.
        \end{enumerate}

        %11
    \item Let $\vec{a},\vec{b},\vec{c}$ be three unit vectors such that $\vec{a}+\vec{b}+\vec{c}=\vec{0}$. If $\lambda=\vec{a}\cdot\vec{b}+\vec{b}\cdot\vec{c}+\vec{c}\cdot\vec{a}$ and $\vec{d}=\vec{a}\times\vec{b}+\vec{b}\times\vec{c}+\vec{c}\times\vec{a}$, then the ordered pair \brak{\lambda, \vec{d}} is equal to: 
        \begin{enumerate}
            \item \brak{-\frac{3}{2}, 3\vec{a}\times\vec{b}}
            \item \brak{\frac{3}{2}, 3\vec{a}\times\vec{c}}
            \item \brak{-\frac{3}{2}, 3\vec{c}\times\vec{b}}
            \item \brak{\frac{3}{2}, 3\vec{b}\times\vec{c}}
        \end{enumerate}

        %12
    \item The coefficient of $x^7$ in the expression $(1 + x)^{10} + x(1 + x)^9 + x^2(1 + x)^8 +\dots+x^{10}$ is :
        \begin{enumerate}
            \item $120$
            \item $210$
            \item $330$
            \item $420$
        \end{enumerate}

        %13
    \item Let $\alpha$ and $\beta$ be the roots of the equation $x^2-x-1=0$. If $p_k=(\alpha)^k+(\beta)^k,k\geq1$, then which one of the following statements is not true ?
        \begin{enumerate}
            \item $p_5=11$
            \item $p_3=p_5-p_4$
            \item $(p_1+p_2+p_3+p_4+p_5)=26$
            \item $p_5=p_2\cdot p_3$
        \end{enumerate}

        %14
    \item The value of $c$ in the Lagrange's mean value theorem for the function $f(x)=x^3-4x^2+8x+11$, when $x\in[0,1]$ is:
        \begin{enumerate}
            \item $\frac{4-\sqrt{7}}{3}$
            \item $\frac{\sqrt{7}-2}{3}$
            \item $\frac{4-\sqrt{5}}{3}$
            \item $\frac{2}{3}$
        \end{enumerate}

        %15
    \item The area (in sq. units) of the region $\cbrak{(x,y)\in\mathbb{R}^2|4x^2\leq y\leq 8x+12}$ is :
        \begin{enumerate}
            \item $\frac{124}{3}$
            \item $\frac{125}{3}$
            \item $\frac{127}{3}$
            \item $\frac{128}{3}$
        \end{enumerate}

\end{enumerate}

\end{document}
