%iffalse
\let\negmedspace\undefined
\let\negthickspace\undefined
\documentclass[journal,12pt,twocolumn]{IEEEtran}
\usepackage{cite}
\usepackage{amsmath,amssymb,amsfonts,amsthm}
\usepackage{algorithmic}
\usepackage{graphicx}
\usepackage{textcomp}
\usepackage{xcolor}
\usepackage{txfonts}
\usepackage{listings}
\usepackage{enumitem}
\usepackage{mathtools}
\usepackage{gensymb}
\usepackage{comment}
\usepackage[breaklinks=true]{hyperref}
\usepackage{tkz-euclide} 
\usepackage{listings}
\usepackage{gvv}                                        
%\def\inputGnumericTable{}                                 
\usepackage[latin1]{inputenc}                                
\usepackage{color}                                            
\usepackage{array}                                            
\usepackage{longtable}                                       
\usepackage{calc}                                             
\usepackage{multirow}                                         
\usepackage{hhline}                                           
\usepackage{ifthen}                                           
\usepackage{lscape}
\usepackage{tabularx}
\usepackage{array}
\usepackage{float}

\newtheorem{theorem}{Theorem}[section]
\newtheorem{problem}{Problem}
\newtheorem{proposition}{Proposition}[section]
\newtheorem{lemma}{Lemma}[section]
\newtheorem{corollary}[theorem]{Corollary}
\newtheorem{example}{Example}[section]
\newtheorem{definition}[problem]{Definition}
\newcommand{\BEQA}{\begin{eqnarray}}
\newcommand{\EEQA}{\end{eqnarray}}
\newcommand{\define}{\stackrel{\triangle}{=}}
\theoremstyle{remark}
\newtheorem{rem}{Remark}

% Marks the beginning of the document
\begin{document}
\bibliographystyle{IEEEtran}
\vspace{3cm}

\title{\textbf{JEE MAINS 2020}}
\author{\textbf{EE24BTECH11019 - Dwarak A}}
\maketitle
\newpage
\bigskip

\renewcommand{\thefigure}{\theenumi}
\renewcommand{\thetable}{\theenumi}

\section*{\textbf{JANUARY 7 SHIFT-2}}
\bigskip

\begin{enumerate}

    \item If $3x+4y=12\sqrt{2}$ is a tangent to the ellipse \brak{\frac{x^2}{a^2}}+\brak{\frac{y^2}{9}}=1 for some $a\in\mathbb{R}$, then the distance between the foci of the ellipse is:
        \begin{enumerate}
            \item $2\sqrt{5}$
            \item $2\sqrt{7}$
            \item $2\sqrt{2}$
            \item $4$
        \end{enumerate}

    \item Let A, B, C and D be four non-empty sets. The Contrapositive statement of "If A $\nsubseteq$ B and B $\nsubseteq$ D then A $\nsubseteq$ C" is :
        \begin{enumerate}
            \item If A $\subseteq$ C, then B $\subset$ A or D $\subset$ B
            \item If A $\nsubseteq$ C, then A $\subseteq$ B and B $\subseteq$ D
            \item If A $\nsubseteq$ C, then A $\nsubseteq$ B and B $\subseteq$ D
            \item If A $\nsubseteq$ C, then A $\nsubseteq$ B or B $\nsubseteq$ D
        \end{enumerate}
    
    \item The coefficient of $x^7$ in the expression $(1 + x)^{10} + x(1 + x)^9 + x^2(1 + x)^8 +\cdots+x^{10}$ is :
        \begin{enumerate}
            \item $420$
            \item $330$
            \item $210$
            \item $120$
        \end{enumerate}

    \item In a workshop, there are five machines and the probability of any one of them to be out of service on a day is $\frac{1}{4}$. If the probability that at most two machines will be out of service on the same day is $\brak{\frac{3}{4}}^3k$, then $k$ is equal to:
        \begin{enumerate}
            \item $\frac{17}{2}$
            \item $4$
            \item $\frac{17}{4}$
            \item $\frac{17}{8}$
        \end{enumerate}

    \item The locus of mid points of the perpendiculars drawn from points on the line $x=2y$ to the line $x=y$ is:
        \begin{enumerate}
            \item $2x-3y=0$
            \item $3x-2y=0$
            \item $5x-7y=0$
            \item $7x-5y=0$
        \end{enumerate}

    \item The value of $\alpha$ for which $4\alpha\int_{-1}^{2}e^{-\alpha\abs{x}}\,dx=5$ is: 
        \begin{enumerate}
            \item $\log_e2$
            \item $\log_e\sqrt{2}$
            \item $\log_e\brak{\frac{4}{3}}$
            \item $\log_e\brak{\frac{3}{2}}$
        \end{enumerate}

    \item If the sum of the first $40$ terms of the series, $3 + 4 + 8 + 9 + 13 + 14 + 18 + 19 + \dots$ is $(102)m$, then $m$ is equal to :
        \begin{enumerate}
            \item $10$
            \item $25$
            \item $5$
            \item $20$
        \end{enumerate}

    \item If $\frac{3+i\sin\theta}{4-i\cos\theta}, \theta\in\sbrak{0,2\pi}$ is a real number, then the argument of $\sin\theta+i\cos\theta$ is :
        \begin{enumerate}
            \item $\pi-\tan^{-1}\brak{\frac{3}{4}}$
            \item $\tan^{-1}\brak{\frac{4}{3}}$
            \item $\pi-\tan^{-1}\brak{\frac{4}{3}}$
            \item $\tan^{-1}\brak{\frac{3}{4}}$
        \end{enumerate}

    \item Let $A=[a_{ij}]$ and $B=[b_{ij}]$ be two $3\times3$ real matrices such that $b_{ij}-(3)^{(i+j-2)}a_{ji}$, where $i,j=1,2,3$. If the determinant of $B$ is $81$, then the determinant of $A$ is :
        \begin{enumerate}
            \item $\frac{1}{9}$
            \item $\frac{1}{81}$
            \item $\frac{1}{3}$
            \item $3$
        \end{enumerate}

    \item Let $f(x)$ be a polynomial of degree $5$ such that $x=\pm1$ are its critical points. If $\lim_{x\to0}\brak{2+\frac{f(x)}{x^3}}=4$ then which of the following is not true ?
        \begin{enumerate}
            \item $f(1)-4f(-1)=4$
            \item $x=1$ is a point of maxima and $x=-1$ is a point of minima of $f$.
            \item $f$ is an odd function
            \item $x=1$ is a point of minima and $x=-1$ is a point of maxima of $f$.
        \end{enumerate}

    \item The number of ordered pairs \brak{r,k} for which $6\cdot\comb{35}{r}=(k^2-3)\cdot\comb{36}{r+1}$, where k is an integer, is :
        \begin{enumerate}
            \item $4$
            \item $6$
            \item $2$
            \item $3$
        \end{enumerate}

    \item Let $a_1,a_2,a_3,\dots$ be a G.P. such that $a_1<0$, $a_1+a_2=4$ and $a_3+a_4=16$. If $\sum_{i=1}^{9}a_i=4\lambda$ then $\lambda$ is equal to : 
        \begin{enumerate}
            \item $171$
            \item $\frac{511}{3}$
            \item $-171$
            \item $-513$
        \end{enumerate}

    \item Let $\vec{a},\vec{b},\vec{c}$ be three unit vectors such that $\overrightarrow{a}+\overrightarrow{b}+\overrightarrow{c}=0$. If $\lambda=\overrightarrow{a}\cdot\overrightarrow{b}+\overrightarrow{b}\cdot\overrightarrow{c}+\overrightarrow{c}\cdot\overrightarrow{a}$ and $\overrightarrow{d}=\overrightarrow{a}\times\overrightarrow{b}+\overrightarrow{b}\times\overrightarrow{c}+\overrightarrow{c}\times\overrightarrow{a}$, then the ordered pair \brak{\lambda, \overrightarrow{d}} is equal to: 
        \begin{enumerate}
            \item \brak{\frac{3}{2}, 3\vec{a}\times\vec{c}}
            \item \brak{\frac{-3}{2}, 3\vec{c}\times\vec{b}}
            \item \brak{\frac{-3}{2}, 3\vec{a}\times\vec{b}}
            \item \brak{\frac{3}{2}, 3\vec{b}\times\vec{c}}
        \end{enumerate}

    \item Let $y=y(x)$ be the solution curve of the differential equation $(y^2-x)\brak{\frac{dy}{dx}}=1$ satisfying $y(0)=1$ This curve intersects the x-axis at a point whose abscissa is :
        \begin{enumerate}
            \item $2+e$
            \item $2$
            \item $2-e$
            \item $-e$
        \end{enumerate}

    \item If $\theta_1$ and $\theta_2$ be respectively the smallest and largest values of $\theta$ in $\brak{0,2\pi}-\cbrak{\pi}$ which satisfy the equation $2\cot^2\theta-\frac{5}{\sin\theta}+4=0$ then $\int_{\theta_1}^{\theta_2}\cos^23\theta\,d\theta$ is equal to:
        \begin{enumerate}
            \item $\frac{2\pi}{3}$
            \item $\frac{\pi}{3}$
            \item $\brak{\frac{\pi}{3}}+\brak{\frac{1}{6}}$
            \item $\frac{\pi}{9}$
        \end{enumerate}

\end{enumerate}

\end{document}
