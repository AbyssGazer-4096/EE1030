\let\negmedspace\undefined
\let\negthickspace\undefined
\documentclass[journal]{IEEEtran}
\usepackage[a5paper, margin=10mm, onecolumn]{geometry}
%\usepackage{lmodern} % Ensure lmodern is loaded for pdflatex
\usepackage{tfrupee} % Include tfrupee package

\setlength{\headheight}{1cm} % Set the height of the header box
\setlength{\headsep}{0mm}     % Set the distance between the header box and the top of the text

\usepackage{gvv-book}
\usepackage{gvv}
\usepackage{cite}
\usepackage{amsmath,amssymb,amsfonts,amsthm}
\usepackage{algorithmic}
\usepackage{graphicx}
\usepackage{textcomp}
\usepackage{xcolor}
\usepackage{txfonts}
\usepackage{listings}
\usepackage{enumitem}
\usepackage{mathtools}
\usepackage{gensymb}
\usepackage{comment}
\usepackage[breaklinks=true]{hyperref}
\usepackage{tkz-euclide} 
\usepackage{listings}
% \usepackage{gvv}                                        
\def\inputGnumericTable{}                                 
\usepackage[latin1]{inputenc}                                
\usepackage{color}                                            
\usepackage{array}                                            
\usepackage{longtable}                                       
\usepackage{calc}                                             
\usepackage{multirow}                                         
\usepackage{hhline}                                           
\usepackage{ifthen}                                           
\usepackage{lscape}
\begin{document}

\bibliographystyle{IEEEtran}
\vspace{3cm}

\title{1.5.37}
\author{EE24BTECH11019 - DWARAK A}
% \maketitle
% \newpage
% \bigskip
{\let\newpage\relax\maketitle}

\renewcommand{\thefigure}{\theenumi}
\renewcommand{\thetable}{\theenumi}
\setlength{\intextsep}{10pt} % Space between text and floats


\numberwithin{equation}{enumi}
\numberwithin{figure}{enumi}
\renewcommand{\thetable}{\theenumi}


\textbf{Question}:
\\
The centre of a circle whose end points of a diameter are \brak{-6, 3} and \brak{6, 4} is \underline{\hspace{1cm}} .

\\
\textbf{Solution: }
\begin{table}[h!]    
  \centering
  \begin{tabular}[12pt]{ |c| c|}
    \hline
    \textbf{Point} & \textbf{Description}\\ 
    \hline
    \vec{A} \brak{-6, 3} & First end-point of the circle's diameter\\
    \hline 
    \vec{B} \brak{6, 4} & Second end-point of the circle's diameter\\
    \hline
    \vec{C} \brak{x, y} & Centre of the circle\\
    \hline   
    \end{tabular}
  \caption{Variables Used}
  \label{tab1.5.37.1}
\end{table}

Center of a circle divides its diameter in the ration $1:1$ internally.\\

Section Formula :
\begin{align}
    \vec{C}=\frac{k\vec{B}+\vec{A}}{k+1}\label{eq1.5.37.1}
\end{align}
Here,
\begin{align}
    k=1\label{eq1.5.37.2}
\end{align}
From  equations \ref{eq1.5.37.1} and \ref{eq1.5.37.2}, the vector $\vec{C}$ is:
\begin{align}
    \vec{C}&=\frac{\vec{A}+\vec{B}}{2}\\
    \vec{C}&=\frac{\myvec{-6\\3}+\myvec{6\\4}}{2}\\
    \vec{C}&=\frac{\myvec{0\\7}}{2}\\
    \vec{C}&=\myvec{0\\3.5}
\end{align}

The coordinates of the center of the circle $\vec{C}$ is $\myvec{0\\3.5}$

\begin{figure}[h!]
   \centering
   \includegraphics[width=1\linewidth]{figs/plot.png}
   \caption{Plot of end-points of diameter $\vec{A}$ and $\vec{B}$ and center of circle $\vec{C}$}
   \label{stemplot}
\end{figure}
\end{document}
\end{document}