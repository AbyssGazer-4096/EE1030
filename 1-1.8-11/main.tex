\let\negmedspace\undefined
\let\negthickspace\undefined
\documentclass[journal]{IEEEtran}
\usepackage[a5paper, margin=10mm, onecolumn]{geometry}
\usepackage{lmodern} % Ensure lmodern is loaded for pdflatex
\usepackage{tfrupee} % Include tfrupee package

\setlength{\headheight}{1cm} % Set the height of the header box
\setlength{\headsep}{0mm}     % Set the distance between the header box and the top of the text

\usepackage{gvv-book}
\usepackage{gvv}
\usepackage{cite}
\usepackage{amsmath,amssymb,amsfonts,amsthm}
\usepackage{algorithmic}
\usepackage{graphicx}
\usepackage{textcomp}
\usepackage{xcolor}
\usepackage{txfonts}
\usepackage{listings}
\usepackage{enumitem}
\usepackage{mathtools}
\usepackage{gensymb}
\usepackage{comment}
\usepackage[breaklinks=true]{hyperref}
\usepackage{tkz-euclide} 
\usepackage{listings}
% \usepackage{gvv}                                        
\def\inputGnumericTable{}                                 
\usepackage[latin1]{inputenc}                                
\usepackage{color}                                            
\usepackage{array}                                            
\usepackage{longtable}                                       
\usepackage{calc}                                             
\usepackage{multirow}                                         
\usepackage{hhline}                                           
\usepackage{ifthen}                                           
\usepackage{lscape}
\begin{document}

\bibliographystyle{IEEEtran}
\vspace{3cm}

\title{1.8.11}
\author{EE24BTECH11019 - DWARAK A}
% \maketitle
% \newpage
% \bigskip
{\let\newpage\relax\maketitle}

\renewcommand{\thefigure}{\theenumi}
\renewcommand{\thetable}{\theenumi}
\setlength{\intextsep}{10pt} % Space between text and floats


\numberwithin{equation}{enumi}
\numberwithin{figure}{enumi}
\renewcommand{\thetable}{\theenumi}


\textbf{Question}:

Find a relation between $x$ and $y$ such that the point $(x, y)$ is equidistant from the point $(3, 6)$ and $(-3, 4)$.

\solution
\begin{table}[h!]    
  \centering
  \begin{tabular}[12pt]{ |c| c|}
    \hline
    \textbf{Point} & \textbf{Description}\\ 
    \hline
    \vec{A} \brak{-6, 3} & First end-point of the circle's diameter\\
    \hline 
    \vec{B} \brak{6, 4} & Second end-point of the circle's diameter\\
    \hline
    \vec{C} \brak{x, y} & Centre of the circle\\
    \hline   
    \end{tabular}
  \caption{Variables Used}
  \label{tab1.8.11.1}
\end{table}

If $\vec{X}$ is equidistant form the points $\vec{A}$ and $\vec{B}$

\begin{align}
	\norm{A-X}&=\norm{B-X}\\
	\norm{A-X}^2&=\norm{B-X}^2\\
	\norm{A}^2-2A^\top X+\norm{X}^2&=\norm{B}^2-2B^\top X+\norm{X}^2\\
	\brak{A-B}^\top X&=\frac{\norm{A}^2-\norm{B}^2}{2}
\end{align}

The above equation is the general expression for the perpendicular bisecting plane between any points $\vec{A}$ and $\vec{B}$.

Substituting the $\vec{A}$ and $\vec{B}$ values in the derived equation.

\begin{align}
	\myvec{6 \\ 2}^\top X &= \frac{\norm{\myvec{3 \\ 6}}^2-\norm{\myvec{-3 \\ 4}}^2}{2}=\frac{20}{2}=10
\end{align}

Comparing with $n^\top x=c$

\begin{align}
	n&=\myvec{6 \\ 2}\\
	c&=10
\end{align}

Line equation :

\begin{align}
	6x+2y&=10
\end{align}

Final line equation :

\begin{align}
	3x+y&=5
\end{align}

\begin{figure}[h!]
   \centering
   \includegraphics[width=0.8\linewidth]{figs/plot.png}
   \caption{Plot of point $\vec{X}$, equidistant from $\vec{A}$ and $\vec{B}$}
\label{Plot}
\end{figure}
\end{document}
\end{document}
