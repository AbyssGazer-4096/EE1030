\let\negmedspace\undefined
\let\negthickspace\undefined
\documentclass[journal]{IEEEtran}
\usepackage[a5paper, margin=10mm, onecolumn]{geometry}
\usepackage{lmodern} % Ensure lmodern is loaded for pdflatex
\usepackage{tfrupee} % Include tfrupee package

\setlength{\headheight}{1cm} % Set the height of the header box
\setlength{\headsep}{0mm}     % Set the distance between the header box and the top of the text

\usepackage{gvv-book}
\usepackage{gvv}
\usepackage{cite}
\usepackage{amsmath,amssymb,amsfonts,amsthm}
\usepackage{algorithmic}
\usepackage{graphicx}
\usepackage{textcomp}
\usepackage{xcolor}
\usepackage{txfonts}
\usepackage{listings}
\usepackage{enumitem}
\usepackage{mathtools}
\usepackage{gensymb}
\usepackage{comment}
\usepackage[breaklinks=true]{hyperref}
\usepackage{tkz-euclide} 
\usepackage{listings}
% \usepackage{gvv}                                        
\def\inputGnumericTable{}                                 
\usepackage[latin1]{inputenc}                                
\usepackage{color}                                            
\usepackage{array}                                            
\usepackage{longtable}                                       
\usepackage{calc}                                             
\usepackage{multirow}                                         
\usepackage{hhline}                                           
\usepackage{ifthen}                                           
\usepackage{lscape}
\begin{document}

\bibliographystyle{IEEEtran}
\vspace{3cm}

\title{4.2.10}
\author{EE24BTECH11019 - DWARAK A}
% \maketitle
% \newpage
% \bigskip
{\let\newpage\relax\maketitle}

\renewcommand{\thefigure}{\theenumi}
\renewcommand{\thetable}{\theenumi}
\setlength{\intextsep}{10pt} % Space between text and floats


\numberwithin{equation}{enumi}
\numberwithin{figure}{enumi}
\renewcommand{\thetable}{\theenumi}


\textbf{Question}:
Find the direction and normal vectors of the line $x-y=2$

\solution
\begin{table}[h!]    
  \centering
  \begin{tabular}[12pt]{ |c| c|}
    \hline
    \textbf{Point} & \textbf{Description}\\ 
    \hline
    \vec{A} \brak{-6, 3} & First end-point of the circle's diameter\\
    \hline 
    \vec{B} \brak{6, 4} & Second end-point of the circle's diameter\\
    \hline
    \vec{C} \brak{x, y} & Centre of the circle\\
    \hline   
    \end{tabular}
  \caption{Variables Used}
  \label{tab4.2.10.1}
\end{table}
\begin{align}
    x-y&=2 \\
    \myvec{1 & -1}\myvec{x \\ y}&=2 \\
    \vec{n}^\top\vec{x}&=c \\
    \implies\vec{n}&=\myvec{1 \\ -1} \\
    \vec{m}^\top\vec{n}&=0 \\
    \myvec{1 & m}\myvec{1 \\ -1}&=0 \\
    1-m&=0 \\
    m&=1 \\
    \implies \vec{m}&=\myvec{1 \\ 1}
\end{align}
The normal vector and direction vector of line $x-y=2$ are $\vec{m}$ and $\vec{n}$ respectively,
\begin{align}
    \vec{n}&=\myvec{1 \\ -1}, \vec{m}=\myvec{1 \\ 1}
\end{align}
\begin{figure}[ht!]
	\centering
   	\includegraphics[width=0.8\linewidth]{figs/fig.png}
   	\caption{Plot of the line, Direction Vector and Normal Vector}
\label{Plot}
\end{figure}
\end{document}
\end{document}
